%                                                                 aa.dem
% AA vers. 9.1, LaTeX class for Astronomy & Astrophysics
% demonstration file
%                                                       (c) EDP Sciences
%-----------------------------------------------------------------------
%%%%%%%%%%%%%%%%%%%%%%%%%%%%%%%%%%%%%%%%%%%%%%%%%%%%%%%%%%%%%%%%%%%%%%%%%%%%%%%
%\documentclass[referee]{aa} % for a referee version
%\documentclass[onecolumn]{aa} % for a paper on 1 column  
%\documentclass[longauth]{aa} % for the long lists of affiliations 
%\documentclass[letter]{aa} % for the letters 
%\documentclass[bibyear]{aa} % if the references are not structured 
%                              according to the author-year natbib style
%%%%%%%%%%%%%%%%%%%%%%%%%%%%%%%%%%%%%%%%%%%%%%%%%%%%%%%%%%%%%%%%%%%%%%%%%%%%%%%
%\documentclass{aa}  
\documentclass[longauth]{aa}
%\documentclass[longauth,referee]{aa}
%%%%%%%%%%%%%%%%%%%%%%%%%%%%%%%%%%%%%%%%%%%%%%%%%%%%%%%%%%%%%%%%%%%%%%%%%%%%%%%
\usepackage{graphicx}
%%%%%%%%%%%%%%%%%%%%%%%%%%%%%%%%%%%%%%%%%%%%%%%%%%%%%%%%%%%%%%%%%%%%%%%%%%%%%%%
\usepackage{txfonts}
%%%%%%%%%%%%%%%%%%%%%%%%%%%%%%%%%%%%%%%%%%%%%%%%%%%%%%%%%%%%%%%%%%%%%%%%%%%%%%%
%\usepackage[options]{hyperref}
% To add links in your PDF file, use the package "hyperref"
% with options according to your LaTeX or PDFLaTeX drivers.
%%%%%%%%%%%%%%%%%%%%%%%%%%%%%%%%%%%%%%%%%%%%%%%%%%%%%%%%%%%%%%%%%%%%%%%%%%%%%%%
% Only include extra packages if you really need them. Common packages are:
\usepackage{graphicx}	% Including figure files
\usepackage{amsmath}	% Advanced maths commands
\usepackage{amssymb}	% Extra maths symbols
\usepackage{lscape}	    % Extra maths symbols
\usepackage{multirow}
\usepackage{xcolor}
\usepackage{ulem}
\usepackage{graphicx}
\usepackage{subfig}
\usepackage{upgreek}
%\usepackage{dirtytalk}
\usepackage{comment}
\usepackage{hyperref}
\usepackage{placeins}
\hypersetup{
    citecolor=blue,
    colorlinks=true,
    linkcolor=blue,
    filecolor=magenta,      
    urlcolor=blue,
    %pdftitle={Overleaf Example},
    pdfpagemode=FullScreen,
    }

\urlstyle{same}
%http://latexcolor.com
\definecolor{amber}{rgb}{1.0, 0.49, 0.0}
\definecolor{violet}{rgb}{0.54, 0.17, 0.89}
%%%%%%%%%%%%%%%%%%%%%%%%%%%%%%%%%%%%%%%%%%%%%%%%%%%%%%%%%%%%%%%%%%%%%%%%%%%%%%%
\graphicspath{{Figures/}}
%%%%%%%%%%%%%%%%%%%%%%%%%%%%%%%%%%%%%%%%%%%%%%%%%%%%%%%%%%%%%%%%%%%%%%%%%%%%%%%
\begin{document} 
   \title{The metal content of the circumgalactic medium around star-forming galaxies at z $\sim$ 2.6 as revealed by the VIMOS Ultra-Deep Survey.}
   \subtitle{}
   \author{M\'endez-Hern\'andez, H.,
          \inst{1}
          %\and
          Cassata, P.\inst{2,3}, %\fnmsep\thanks{Just to show the usage
          %of the elements in the author field}
          Ibar, E.\inst{1},
          Amor\'in, R. \inst{4,5},
          Aravena, M. \inst{6},
          Bardelli, S. \inst{7},
          Cucciati, O. \inst{7},
          Garilli, B. \inst{8},
          Giavalisco, M. \inst{9},
          Guaita, L. \inst{10},
          Hathi, N. \inst{11},
          Koekemoer, A. \inst{11},
          Le Brun, V. \inst{12},
          Lemaux, B.C. \inst{13,14},
          Maccagni, D. \inst{8},
          Ribeiro, B. \inst{12},
          Tasca, L. \inst{12},
          Tejos, N. \inst{16},
          Thomas, R. \inst{17},
          Tresse, L. \inst{12},          
          Vergani, D. \inst{7},
          Zamorani, G. \inst{7},
          Zucca, E.  \inst{7}
          }
   \institute{Insituto de F\'isica y Astronom\'ia, Universidad de Valpara\'iso,
              Avda. Gran  Breta\~na 1111, 2340000 Valpara\'iso, Chile\\
              \email{hugo.mendez@postgrado.uv.cl}
         \and
             Dipartimento di Fisica e Astronomia Galileo Galilei, Universit\`a 
             degli Studi di Padova Vicolo del $L'$Osservatorio 3, 35122 Padova Italy
             %\email{paolo.cassata@unipd.it}
        \and
             INAF Osservatorio Astronomico di Padova, vicolo dell'Osservatorio 5, I-35122 Padova,Italy
        \and
            Instituto de Investigaci\'on Multidisciplinar en Ciencia y Tecnolog\'ia, Universidad de La Serena, Ra\'ul Bitr\'an, 1305 La Serena, Chile
        \and
            Departamento de Astronom\'ia, Universidad de La Serena, Av. Juan Cisternas 1200 Norte, La Serena, Chile             
        \and
            N\'ucleo de Astronom\'ia, Facultad de Ingenier\'ia y
            Ciencias, Universidad Diego Portales, Av. Ej\'ercito 441, Santiago, Chile
        \and 
            INAF-Osservatorio di Astrofisica e Scienza dello Spazio di Bologna, 
            via Gobetti 93/3, 40129 Bologna Italy
        \and
            INAF-IASF, via Bassini 15, I-20133, Milano, Italy
        \and
            Department of Astronomy, University of Massachusetts Amherst, 710 N. Pleasant St, Amherst, MA 01003, USA
        \and
            Departamento de Ciencias Fisicas, Facultad de Ciencias Exactas, 
            Universidad Andres Bello, Fernandez Concha 700, Las Condes, Santiago, Chile
        \and
            Space Telescope Science Institute, 3700 San Martin Drive, Baltimore, MD 21218, USA
        \and
            Aix Marseille Univ, CNRS, CNES, LAM, Marseille, France
        \and
            Gemini Observatory, NSF's NOIRLab, 670 N. A'ohoku Place, Hilo, Hawai'i, 96720, USA
        \and
            Department of Physics and Astronomy, University of California, Davis, One Shields Ave., Davis, CA 95616, USA
        \and
            Space Telescope Science Institute, 3700 San Martin Drive, Baltimore, MD 21218, USA
        \and
            Instituto de F\'isica, Pontificia Universidad Cat\'olica de Valpara\'iso, Casilla 4059, Valpara\'iso,
            Chile
        \and
            European Southern Observatory, Av. Alonso de C\'ordova 3107, Vitacura, Santiago, Chile
            }
   \date{Received XXXX XX, 2021; accepted XXXX XX, 2022}
%%%%%%%%%%%%%%%%%%%%%%%%%%%%%%%%%%ABSTRACT%%%%%%%%%%%%%%%%%%%%%%%%%%%%%%%%%%%%%
% \abstract{}{}{}{}{} 
% 5 {} token are mandatory
  \abstract
  % context heading (optional)
  % {} leave it empty if necessary  
   {The circumgalactic medium (CGM) is the location where the interplay
   between large-scale outflows and accretion onto galaxies occurs. Metals in
   different ionization states flowing between the circumgalactic and
   intergalactic mediums are affected by large galactic outflows and
   low-ionization state inflowing gas. Observational studies on their spatial
   distribution and their relation with galaxy properties may provide
   important constraints on models of galaxy formation and evolution.}
  % aims heading (mandatory)
   {The main goal of this paper is to provide new insights into the spatial
   distribution of the circumgalactic of star-forming galaxies at $1.5 < z <
   4.5$ ($\langle z\rangle\sim$2.6) in the peak epoch of cosmic
   star formation activity in the Universe. We also look for possible
   correlations between the strength of the low- and high-ionization
   absorption features(LIS and HIS) and stellar mass, star formation rate,
   effective radius, and azimuthal angle $\phi$ that defines the location of
   the absorbing gas relative to the galaxy disk plane.}
  % methods heading (mandatory)
   {The CGM has been primarily detected via the absorption features that it
   produces on the continuum spectrum of bright  \textit{background} sources.
   We select a sample of 238 close pairs from the VIMOS Ultra Deep Survey to
   examine the spatial distribution of the gas located around star-forming
   galaxies and generate composite spectra by co-adding spectra of \textit
   {background} galaxies that provide different sight-lines across the CGM of
   star-forming galaxies.}
  % results heading (mandatory)
   {We detect LIS (C\,{\sc ii} and Si\,{\sc ii}) and HIS (Si\,{\sc iv}, C\,
   {\sc iv}) up to separations $\langle b \rangle=$\,172\,kpc and 146\,kpc.
   Beyond this separation, we do not detect any significant
   signal of CGM absorption in the \textit{background} composite spectra. Our
   Ly$\upalpha$, LIS and HIS rest-frame equivalent width $W_{0}$ radial
   profiles are at the upper envelope of the $W_{0}$ measurements at lower
   redshifts, suggesting a potential redshift evolution for the CGM gas
   content producing these absorptions. We find a correlation between C\,
   {\sc ii} and C\,{\sc iv} with star formation rate, stellar mass as well as
   trends with galaxy size estimated by the effective radius and azimuthal
   angle. Galaxies with high star formation rate (log[SFR/(M$_{\odot}$yr$^
   {-1})]>1.5$) and stellar mass (log[M$_{\bigstar}$/M$_{\odot}$]>10.2) show
   stronger C\,{\sc iv} absorptions compared with those low SFR (log[SFR/(M$_
   {\odot}$yr$^{-1})]<0.9$) and low stellar mass (log[M$_{\bigstar}$/M$_
   {\odot}$]<9.26). The latter population instead shows stronger  C\,
   {\sc ii} absorption than their more massive or more star-forming
   counterparts. We compute the C\,{\sc ii} / C\,{\sc iv} $W_{0}$ line ratio
   that confirms the C\,{\sc ii} and C\,{\sc iv} correlations with impact
   parameter, stellar mass and star formation rate. We do not find any
   correlation with $\phi$ in agreement with other high-redshift studies and
   in contradiction to what is observed at low-redshift where  large-scale
   outflows along the minor axis forming bipolar outflows are detected.}
  % conclusions heading (optional), leave it empty if necessary 
   {We find that the stronger C\,{\sc iv} line absorptions in the outer
   regions of these {star-forming} galaxies could be explained by stronger
   outflows in galaxies with higher star formation rates and stellar masses,
   capable of projecting the ionized gas up to large distances, and/or by
   stronger UV ionizing radiation in these galaxies, able to ionize the gas
   even at large distances. On the other hand, low-mass galaxies show
   stronger C\,{\sc ii} absorptions, suggesting larger reservoirs of cold gas
   that could be explained by a softer radiation field unable to ionize
   high-ionization state lines or by the galactic fountain scenario where
   metal-rich gas ejected from previous star-formation episodes fall back to
   the galaxy. These large reservoirs of cold neutral gas around low-mass
   galaxies could be funnelled into the galaxies and eventually provide the
   necessary fuel to sustain star formation activity.}
%%%%%%%%%%%%%%%%%%%%%%%%%%%%%%%%%%%%%%%%%%%%%%%%%%%%%%%%%%%%%%%%%%%%%%%%%%%%%%%
   \keywords{galaxies: evolution --
            galaxies: high-redshift --
            galaxies:ISM --
            galaxies: CGM}

   \titlerunning{The CGM metal content around star-forming galaxies at $z\sim$2.6 revealed by VUDS}
   \authorrunning{M\'endez-Hern\'andez, H.}
   \maketitle
%%%%%%%%%%%%%%%%%%%%%%%%%%%%%%%%%%ABSTRACT%%%%%%%%%%%%%%%%%%%%%%%%%%%%%%%%%%%%%
%
%------------------------------------------------------------------------------
%%%%%%%%%%%%%%%%%%%%%%%%%%%%%%%%%%%%%%%%%%%%%%%%%%%%%%%%%%%%%%%%%%%%%%%%%%%%%%%
%%%%%%%%%%%%%%%%%%%%%%%%%%%%%%%%%%%%INTRO%%%%%%%%%%%%%%%%%%%%%%%%%%%%%%%%%%%%%%
%%%%%%%%%%%%%%%%%%%%%%%%%%%%%%%%%%%%%%%%%%%%%%%%%%%%%%%%%%%%%%%%%%%%%%%%%%%%%%%
\section{Introduction}
\noindent
The circumgalactic medium (CGM, 10kpc-$\sim$300 kpc) is the gas reservoir
between the interstellar medium (ISM, {$\lesssim10$kpc}) and the
intergalactic medium (IGM, $\gtrsim300$kpc). It is characterized as an active
interface where galaxies reprocess their baryonic material. In fact, up to
$\sim$50\% of the total baryonic mass is found in the CGM at both,
low-redshift ($z\sim0.2$) \citep{Wolfe05,Werk14,Zheng15} and
high-redshift \citep{Hafen19}, thus representing a significant gas reservoir
that can, for example, feed the ISM with gas to form new stars \citep
{Zhu13a,Thom12,Richter16}. Supernovae explosions or strong stellar winds can
deposit metals in the surrounding medium, gas that is mixed with pristine gas
accreted from the intergalactic medium (IGM; \citealt
{Prochaska09,Bauermeister10,Tumlinson11a,Tumlinson17,Kacprzak17}).

Since the first discoveries of circumgalactic gas around star-forming
galaxies \citep{Boksenberg78,Kunth84,Bergeron86}, its study by direct
detection of emission has been a challenge \citep
{vandeVoort13,Burchett21}. The CGM has been primarily detected via the
absorption features that it produces on light from \textit
{background} sources. In a pioneering study at high-redshift
($z\gtrsim 1.5$), 
\cite{Adelberger03,Adelberger05a} studied the dependence of the absorption
strength on the projected angular separation between \textit
{foreground} and \textit{background} galaxies(i.e., impact parameter $b$),
reporting absorption detection at impact parameters up to 40\,kpc and
noticed that absorption strength weakens significantly at larger
separations. Thanks to their brightness, quasars are typically used
as \textit{background} sources to probe the CGM around \textit
{foreground} galaxies
\citep{Bergeron91,Steidel94,Kacprzak10}, allowing to identify the signal
produced by extremely low column densities (N$_{\rm{HI}}\simeq10^{12}$ cm$^
{-2}$), independent of the properties of the targeted galaxy
(e.g.,  luminosity and/or redshift) 
\citep{Tumlinson17,Peroux20a}. This technique has been successful to probe the
CGM of galaxies up to  redshifts $z\sim$5 \citep{Matejek12} and has led to
the discovery that the CGM is gas rich and has a multi-phase nature
\citep
{Adelberger03,Steidel94,Steidel95,Songaila01,Tripp98,Kacprzak10,Werk16}. 

The CGM has been also characterized using the so-called "down-the-barrel
technique", that uses the targeted galaxy itself as a \textit
{background} source. Absorption lines redshifted with respect to the galaxy
rest-frame evidence the presence of gas flowing towards the galaxy. This
result has been interpreted as a strong indication of inflowing material onto
the host galaxy \citep{Sato09,Rubin12,Martin12}, while outflows have been
identified as blueshifted absorptions in galaxy spectra \citep
{Martin05,Chen10}. Although the location of the gas producing the detected
absorption is unconstrained, this technique has been successful in studying
galactic inflows and outflows from the spectroscopy of star-forming galaxies
up to $z\sim2-3$ \citep{Kornei12,Rudie12a,Heckman15,Heckman16}. 

Recently three more techniques have been used to explore the CGM. The first
one takes advantage of strong gravitational lensing, the so-called
"gravitational-arc tomography", that uses giant bright lensed arcs as
background sources to map the CGM of \textit{foreground} galaxies providing
a tomographic view of the absorbing gas \citep
{Lopez18,Claeyssens19,Lopez20,Tejos21,Mortensen21}. The second one takes
advantage of deep three-dimensonal datacubes observations to study the cold
CGM of high-redshift ($z > 2$) star-forming galaxies, and has reported the
ubiquitous presence of Ly$\upalpha$ haloes in these galaxies 
\citep{Steidel11,Matsuda12,Momose14,Wisotzki16,Leclercq17,Chen20}, whose line
properties are correlated to their spatial location 
\citep{Leclercq20} and that can extend up to 4Mpc beyond the CGM 
\citep{Chen20,Bacon21}. The third one uses the spectra of bright afterglows of
long gamma-ray bursts (GRBs) to derive the kinematic properties of the CGM
gas around their host galaxies and constrain the physical properties 
\citep{Gatkine19,Gatkine22}.

Alongside these diverse techniques to study the CGM, the large spectroscopic
extragalactic surveys have allowed for the statistical extraction and
analysis of the weak signals from different metal absorption using the
stacking of hundreds of spectra \citep{Steidel10}. These large datasets have
been helpful  to overcome the limitations of finding galaxy-QSOs pairs at
concordant redshifts \citep{Steidel94,Bouche07}, increasing statistics and
facilitating vast parameter space exploration 
\citep{York06,Bordoloi11,Zhu13a}. At low-redshift ($z_{\rm{med}}\sim$0.5) CGM
analyses have been focused on the study of Ly$\upalpha$ \citep{Chen01b}, C\,
{\sc iv} \citep{Chen01a}, O\,{\sc vi} \citep{Tumlinson11a}, Mg\,
{\sc ii} \citep{Bowen95,Bouche07,Steidel94} absorption lines, and their
dependence with stellar mass \citep{Bordoloi11}, inclination
\citep{Kacprzak10}, and azimuthal angle \citep{Shen12,Bordoloi14a}. At higher
redshifts, the observed metal absorption lines are mostly limited to Si\,
{\sc ii}, C\,{\sc ii}, C\,{\sc iv}, and Si\, {\sc iv}, where the first two
are called low-ionization state (LIS, T=$10^{4-4.5}$K) lines, while the last
two high-ionization state (HIS, T=$10^{4.5-5.5}$K) lines \citep
{Steidel10}. At high redshift (z>2) several authors have reported the
presence of high-velocity outflows. \cite{Lehner14} demonstrated that O\,
{\sc vi} successfully probe outflows in star-forming galaxies and reported
velocity widths in the range 200-400kms$^{-1}$. Next, \cite{Du16} analysed
several ionization lines (e.g. Si\,{\sc ii}, Fe\,{\sc ii}, Al\,{\sc ii},
Ni\,{\sc ii}, Al\,{\sc iii}, C\,{\sc iv}) to probe the multiphase nature of
the CGM and detected C\,{\sc iv} blueshifted offsets concordant  with
76kms$^{-1}$ velocity outflows. Their results show a direct link between C\,
{\sc iv} absorption and star formation rate. Later on, \cite
{Jones18} reported velocity outflows of $\sim$150km$^{-1}$ as shown by
several LIS and HIS absorptions in nine gravitationally lensed star-forming
galaxies($z\simeq2-3$), suggesting that galaxy outflows regulate the galaxy
chemical evolution. Similar outflows detections inferred from different
ionization absorption lines have been reported, and where stellar mass and
star-formation rate have been invoked as their main drivers \citep
{Zhu13a,Turner14,Trainor15,Gatkine19,Price20}. However, as suggested
by \cite{Dutta21} LIS and HIS absorption detections (from which velocity
outflows can be inferred), could be affected by large-scale environmental
processes \citep{Dutta21,WangS22} or their available neutral gas
content \citep{Berry12,Oyarzun16,Du18}, and their correlation with stellar
mass and star-formation rate, in fact, might be a consequence of a
main-sequence offset, rather than simply correlated with their
star-formation rate or stellar mass \citep{Cicone16,Gatkine22}.

Numerical simulations consider that the CGM comes from gas accreted from the
intergalactic medium (IGM), followed by stellar winds from the central galaxy
and the gas ejected or stripped from satellites \citep
{Hafen19,Christensen18}. Simulations reveal that the accretion efficiency
depends on the galaxy stellar mass, decreasing from $\sim$80 per cent for M$_
{\bigstar} \sim 10^{6}$M$_{\odot}$ galaxies to $\sim$60 per cent for M$_
{\bigstar} \sim 10^{10}$M$_{\odot}$ galaxies
\citep{Dekel05,Keres05,Hafen20}. Once accreted, this material can remain in
the CGM for billions of years leading to a well mixed-halo gas before it
interacts with the ejected large-scale stellar winds produced by starbursts
\citep{Martin05,Weiner09}. At $z=2$, most metals are found to be located in
the interstellar medium (ISM) or stars of the central galaxy and by $z=0.25$
most of it  will end up in the CGM and IGM(e.g., \citealt
{Peeples14,Angles17,Hafen20,Oppenheimer16,Nelson21}.)

The way in which this pristine material is accreted into galaxies may depend
on its location relative to the galaxy disk plane defined by the azimuthal
angle ($\phi$). Various observational studies highlight a correlation between
the strength of the Mg\,{\sc ii} LIS absorption line and the azimuthal
angle 
\citep{Bordoloi11,Bordoloi14a,Kacprzak11b,Bouche12,Bouche13}. Although weak
Mg\,{\sc ii} detections along the minor axis have been reported
(e.g.,  \citealt{Lan14}), it has been found that strong Mg\,{\sc ii}
absorptions are preferably detected along the major-axis of galaxies,
suggesting the presence of inflowing material potentially feeding future
star formation. On the other hand, C\,{\sc iv} and O\,{\sc vi} (HIS lines)
absorptions seem stronger along the minor axis, probably evidencing strong
stellar winds enriching the CGM 
\citep{Tumlinson11a,Kacprzak15a}. Recently, 
\cite{Peroux20b} used cosmological hydrodynamical simulations to examine the
physical properties of the gas located in the CGM of star-forming galaxies
($z<1$) as a function of angular orientation. They found that the CGM
properties vary strongly with the impact parameter, stellar mass, and
redshift. They reported a higher average CGM metallicity at large impact
parameters ($b>$100kpc) along the minor versus major axes. Moreover, they
tentatively found that the average metallicity of the CGM depends on the
azimuthal angle, showing that the low metallicity gas is preferably
inflowing along the galaxy major axis, while outflows are commonly located
along the minor axis, concordant with previous observations 
\citep{Bordoloi11,Bouche12,Kacprzak12a,Kacprzak15a}. These results frame a
picture where star-forming galaxies accrete co-planar gas within narrow
stream-flows providing fresh fuel for the new generation of stars; later,
this population will produce metal-enriched galactic-scale outflows along
the minor axis \citep{Kacprzak17}. However, even though the presence of
bipolar outflows collimated along the minor axis is expected to evolve with
redshift (expected to be ubiquitous at z = 1),  it has been difficult to
demonstrate its presence at z = 2, mostly as the result of the absence of gaseous
galactic discs sculpting the outflows \citep{Nelson19}.

In order to better understand the mechanisms of galaxy growth in earlier
phases of the history of the Universe, and to investigate a possible cosmic
evolution of such mechanisms, similar studies at higher redshifts are needed.
Observations along different lines of sight (l.o.s.) can probe different
parts of the CGM, and the identification of low and high ionization metal
absorption lines can give information on the possible multi-phase nature of
the CGM. However, using QSO-galaxy pairs to study the CGM, is not
straightforward, as the identification of the galaxies responsible for the
metal absorptions detected on the spectra of background quasars, is not an
easy task. Stacking analyses of \textit{background} galaxy spectra that are
located in the vicinity of \textit{foreground} galaxies provide an
alternative tool to overcome sensitivity limitations, exploiting large
extra-galactic surveys containing a large number of individual spectra. In
this work, we seek to characterize the presence of low- (LIS: \ O\,
{\sc i}+Si\,{\sc ii}, C\,{\sc ii}, Si\,{\sc ii}, Fe\,{\sc ii}),
intermediate- (IIS: Al\,{\sc iii}) and high-ionization (HIS: Si\,{\sc iv},
C\,{\sc iv}) state metal absorption (see Table \ref{Tbl:Lines}) in the CGM of
a star-forming galaxy population at $\langle z \rangle$ $\sim$ 2.6 using UV
spectra obtained from the large VIMOS Ultra Deep Survey (VUDS; \citealt
{LeFevre15}). We use thousands of galaxies with the most reliable redshift
measurements and the highest S/N spectra (with reliability flags 3 and 4;
see \citealt{LeFevre15,Tasca17}), a sample broadly representative of the bulk
of the star-forming galaxy population at these redshifts \citep{Lemaux22}. To
detect the dim signal coming from low- and high ionization line absorptions
produced in the CGM of these star-forming galaxies, we stack the spectra of
close (in projection) \textit{background} galaxies to establish the metal
distribution in the CGM and explore their dependence on the physical and
morphological properties of star-forming galaxies (e.g.,  impact parameter
$b$, star formation rate, stellar mass, galaxy effective radius and azimuthal
angle) at the peak epoch of cosmic star formation activity in the Universe. 

The manuscript is organized as follows. Section~\ref{sec:Data} summarizes the
VUDS survey properties relevant to our analyses. Section~\ref
{sec:Ana} presents our stacking  analysis method for measuring the metal
equivalent widths ($W_{0}$), while Section~\ref{sec:Res} presents our $W_
{0}$ results for Ly$\upalpha$, and multiple LIS, IIS and HIS metal lines
observed our star-forming galaxy sample across different physical and
morphological properties. We discuss our results in Section~\ref
{sec:Dis}, and finally we present our conclusions in Section~\ref
{sec:Con}. Throughout the text, we use $\Lambda$CDM cosmology with H$_
{0}$\,=\,70\,km\,s$^{-1}$\,Mpc$^{-1}$, $\Omega_{\rm M}$=0.3 and $\Omega_
{\Lambda}$=0.7, and distances are given in physical units (kpc).
%%%%%%%%%%%%%%%%%%%%%%%%%%%%%%%%%%%%%%%%%%%%%%%%%%%%%%%%%%%%%%%%%%%%%%%%%%%%%%%
%%%%%%%%%%%%%%%%%%%%%%%%%%%%%%%%%%%%INTRO%%%%%%%%%%%%%%%%%%%%%%%%%%%%%%%%%%%%%%
%%%%%%%%%%%%%%%%%%%%%%%%%%%%%%%%%%%%%%%%%%%%%%%%%%%%%%%%%%%%%%%%%%%%%%%%%%%%%%%

%%%%%%%%%%%%%%%%%%%%%%%%%%%%%%%%%%%%%%%%%%%%%%%%%%%%%%%%%%%%%%%%%%%%%%%%%%%%%%%
%%%%%%%%%%%%%%%%%%%%%%%%%%%%%%%%%%%%%DATA%%%%%%%%%%%%%%%%%%%%%%%%%%%%%%%%%%%%%%
%%%%%%%%%%%%%%%%%%%%%%%%%%%%%%%%%%%%%%%%%%%%%%%%%%%%%%%%%%%%%%%%%%%%%%%%%%%%%%%
\section{The VUDS parent sample}\label{sec:Data}

VUDS has obtained spectra of 5590 galaxies in the redshift range $1.5 < z <
4.5$. A detailed description  of the survey observations,
the methods applied to process the data, and the derived parameters including
the spectroscopic redshifts $z_{\rm{spec}}$ is given in \cite{LeFevre15}. A
description of the VUDS-DR1 first data release can be found in \cite
{Tasca17}. The VUDS spectroscopic targets are selected based on their
photometric  redshifts and observed optical flux; the targets have $z_
{phot}$ + 1$\sigma_{\rm z} \geq$ 2.4 and  i$_{AB} \leq$ 25. The  wavelength
range of each spectrum is between 3600\,$<\uplambda/\AA<$\,9350, accumulating
14\,hrs of integration time in each of the LRBLUE and LRRED grisms of the
VIMOS spectrograph on the ESO Very Large Telescope \citep{LeFevre03}, with a
spectral resolution R\,$\sim$\,230($\sim 7\,\AA$) and reaching a S/N=5 on the
continuum at 8500\,$\AA$. We note that the VUDS observations were taken using
the VIMOS low-resolution multi-slit mode with a minimum slit length optimized
to 6\,arcsec, maximizing the number of observed slits per mask (see \citealt
{Bottini05}).

Standard data processing was performed using the VIPGI environment  
\citep{Scodeggio05}, followed by redshift measurements using the {\sc EZ}
package \citep{Garilli10}. The final UV rest-frame flux-limited sample is
broadly representative of the bulk of the star-forming galaxy population at
these redshifts ($2 < z < 5$) \citep{Lemaux22}. A critical aspect of VUDS is
the large comoving volume covered, totalling of  1\,deg$^{2}$ in three
fields: COSMOS \citep{Scoville07}, ECDFS \citep{Giacconi02}, and
VVDS02h \citep{LeFevre05b,LeFevre13b}.

For the purposes of this study, the instrumental setup translates into the
ability to follow lines redder than Ly$\upalpha$ ($\geq1215.6\AA$)  at
redshifts higher than 1.5. We decided to restrict the VUDS sample to 2100
galaxies, selected in the redshift range $1.5 < z < 4.5$ ($z_{\rm
{flag}}=3, 4$, i.e.\ $95-100\%$ probability  of being correct). We note that
the velocity accuracy of redshift measurements is expected to be in the range
d$z$ / (1 + $z$) = 0.0005-0.0007, or an absolute velocity ($\sigma^{-1}_
{v}$) accuracy 150-200 km  \citep{LeFevre13b,LeFevre15,Tasca17}. The rich
broadband imaging available for galaxies in VUDS is used for spectral energy
distribution (SED) fitting \citep{Tasca15,Thomas17} using the Galaxy
Observed-Simulated SED Interactive Program ({\sc GOSSIP};
\citealt{Franzetti08}) to derive various global galaxy properties
(e.g., stellar mass, SFR, dust extinction E(B-V), age, or metallicity).

{\sc GOSSIP} is a tool that performs SED fitting by using a combination of
spectro-photometric measurements from different bands to match a set of
synthetic galaxy spectra based on emission from stellar populations.
{\sc GOSSIP} uses model spectra from galaxy population synthesis models 
\citep{Bruzual03,Maraston05} and uses the probability distribution function
(PDF) of each galaxy parameter to determine the best SED fit. Figure \ref
{Fig:SFR-STM} shows SFR versus stellar mass for 5590 star-forming galaxies
at: $z > 1.5$ selected from the VUDS survey (gray squares), $2 < z < 3.5$
(yellow crosses) and the 238 star-forming galaxies (blue stars) that we
finally selected and scrutinized in this work (see Section~\ref
{sec:Ana}). These 238 galaxies have stellar masses and star formation rates
of log[M$_{\bigstar}$/M$_{\odot}$])=$9.73\pm0.4$ and log[SFR/(M$_
{\odot}$yr$^{-1}$)]=$1.38\pm0.39$.\\

%%%%%%%%%%%%%%%%%%%%%%%%%FIG-Scatter-HISTO-RedShift%%%%%%%%%%%%%%%%%%%%%%%%%%%%
\begin{figure}
\centering
\includegraphics[width=8.5cm]{ScatterHisto-mass-SFR.pdf}
\caption{SFR versus stellar mass for 5590 star-forming galaxies selected from
the  VUDS survey, at $z>1.5 $ (gray open squares), $2 < z < 3.5$
(yellow crosses) and the 238 star-forming galaxies(blue stars) considered in
this work. The black dot-dashed line shows the main sequence of star-forming
galaxies(SFG) for $1.5 < z < 2.5$ including its $\pm.3$dex scatter
represented by the green shaded region, as defined by \citealt{Hathi16}. The
top and right panels report the distributions in stellar mass and SFR,
for the 238 star-forming galaxies evaluated in this work.}
\label{Fig:SFR-STM}%
\end{figure}  
%%%%%%%%%%%%%%%%%%%%%%%%%FIG-Scatter-HISTO-RedShift%%%%%%%%%%%%%%%%%%%%%%%%%%%%

Exploiting the Hubble Space Telescope (HST) $F814W$ images \citep
{Koekemoer07} that are available as part of the COSMOS Survey \citep
{Scoville07}, morphological parameters have been estimated for 1242
star-forming galaxies($z_{\rm{flag}}=4, 3, 2, 9$, i.e.\ $75\%$ probability of
being correct) with 9.5<log[M$_{\bigstar}$/M$_{\odot}$]<11.5. In
particular, \cite{Ribeiro16} run {\sc Galfit} \citep{Peng02,Peng10}, a
standard parametric profile-fitting tool, estimating Sersic indices
($n$), azimuthal angles, major to minor axis ratios ($q$) and effective
radii ($r_{\rm eff}$). Figure~\ref{Fig:PairsHisto} shows the distribution of
the 97 selected galaxy pairs (see Section~\ref{sec:Ana}) with available
morphological parameters. Table \ref{Tbl:Samples} shows the detailed number
of subsets of galaxies.

%%%%%%%%%%%%%%%%%%%%%%%%%%%%%%%%%%%FIG-HISTO%%%%%%%%%%%%%%%%%%%%%%%%%%%%%%%%%%%
\begin{figure*}
\centering
\includegraphics[width=17cm]{MultiPrp-Histogram-VUDS-SubsetsRows-kpcC.pdf}
\caption{Projected angular separation ($\theta$) amongst galaxy 
\textit{foreground-background} galaxy pairs, axial ratio (q), 
effective radius (r$_{\rm{eff}}$) and azimuthal angle ($\phi$) distributions
of the star-forming (\textit{foreground}) galaxies used in our analyses. 
Morphological parameters were obtained from 
the parametric measures of \cite{Ribeiro16}, 
see section \ref{sec:Data} for details.}
\label{Fig:PairsHisto}%
\end{figure*}  
%%%%%%%%%%%%%%%%%%%%%%%%%%%%%%%%%%%FIG-HISTO%%%%%%%%%%%%%%%%%%%%%%%%%%%%%%%%%%%

%%%%%%%%%%%%%%%%%%%%%%%%%%%%%%%%%%%TBL-LINES%%%%%%%%%%%%%%%%%%%%%%%%%%%%%%%%%%%
\begin{table}
\centering
\caption{\label{Tbl:Samples}
VUDS \textit{fg-bg} galaxy pair sample.}
\begin{tabular}{cc}
  \\
  \hline\hline
  \multicolumn{1}{c}{Sample} &
  \multicolumn{1}{c}{N} \\
  \hline\hline  
Galaxies $z > 1.5$, $z_{\rm{flag}}$=3,4,2,9 \tablefootmark{a}  & 5590 \\ %1740
Galaxies $z > 1.5$, $z_{\rm{flag}}$=3,4  \tablefootmark{b} & 2100 \\
\textit{fg-bg} galaxy pairs, $z_{\rm{flag}}$=3,4 \tablefootmark{c} & 238 \\
Galaxies fitted with {\sc Galfit}, $z_{\rm{flag}}$=3,4,2,9  \tablefootmark{d} & 1242 \\
\textit{fg-bg} galaxy pairs, $z_{\rm{flag}}$=3,4 + {\sc Galfit}  \tablefootmark{e} & 97 \\
\hline
\end{tabular}
\tablefoot{
Number of objects of our \textit{fg-bg} galaxy pair sample sample
(see Sect. \ref{sec:Data} for details).\\
\tablefoottext{a}{VUDS parent sample.}
\tablefoottext{b}{Star-forming galaxies at $z>1.5$.}
\tablefoottext{c}{Star-forming galaxies with morphological parameters available.}
\tablefoottext{d}{Close \textit{fg-bg} galaxy pairs.}
\tablefoottext{e}{Close \textit{fg-bg} galaxy pairs with morphological 
parameters.}}
\end{table}
%%%%%%%%%%%%%%%%%%%%%%%%%%%%%%%%%%%TBL-LINES%%%%%%%%%%%%%%%%%%%%%%%%%%%%%%%%%%%

%%%%%%%%%%%%%%%%%%%%%%%%%%%%%%%%%%%%%%%%%%%%%%%%%%%%%%%%%%%%%%%%%%%%%%%%%%%%%%%
%%%%%%%%%%%%%%%%%%%%%%%%%%%%%%%%%%%%%DATA%%%%%%%%%%%%%%%%%%%%%%%%%%%%%%%%%%%%%%
%%%%%%%%%%%%%%%%%%%%%%%%%%%%%%%%%%%%%%%%%%%%%%%%%%%%%%%%%%%%%%%%%%%%%%%%%%%%%%%

%%%%%%%%%%%%%%%%%%%%%%%%%%%%%%%%%%%%%%%%%%%%%%%%%%%%%%%%%%%%%%%%%%%%%%%%%%%%%%%
%%%%%%%%%%%%%%%%%%%%%%%%%%%%%%%%%%%ANALYSIS%%%%%%%%%%%%%%%%%%%%%%%%%%%%%%%%%%%%
%%%%%%%%%%%%%%%%%%%%%%%%%%%%%%%%%%%%%%%%%%%%%%%%%%%%%%%%%%%%%%%%%%%%%%%%%%%%%%%
\section{Analysis}\label{sec:Ana}

In this work, we seek to probe the CGM around galaxies at  $z \sim$ 2.6, near
the epoch at which the cosmic star formation rate density of the Universe
reaches its peak, by using the spectra of \textit{background (bg)} galaxies
as "spotlights" illuminating the CGM. At these redshifts, individual \textit
{bg} galaxies are not bright enough to individually detect low- and
high-ionization state metal absorption lines. Thus, we stack spectra
of \textit{bg} galaxies to look for absorption lines produced by the CGM
around \textit{foreground (fg)} star-forming galaxies. To select a sample of
close galaxy pairs(\textit{fg-bg}) out of the VUDS survey, we have imposed
similar criteria as those described by \cite {Steidel10}: (i) galaxy spectra
for the \textit{fg} and \textit{bg} galaxies must possess an accurately 
determined redshift ($95-100\%$ see \citealt{LeFevre15,Tasca17} for details), 
(ii) a redshift separation 0.1 $< z_{\rm{bg}}$ - $z_{\rm{fg}} \leqslant$ 1.0
between \textit{fg-bg} galaxy pairs to ensure that each spectrum contains a
significant common spectral coverage after shifting to the rest-frame of
the \textit {fg} galaxy and avoid an overlap between the $fg$ CGM and the
$bg$ galaxy absorptions., and iii) a maximum projected angular separation of
23$\arcsec$.

%%%%%%%%%%%%%%%%%%%%%%%%%%%%%%%FIG_HISTO-RedShift%%%%%%%%%%%%%%%%%%%%%%%%%%%%%%
\begin{figure}
\centering
\includegraphics[width=8.5cm]{MultiPrp-Histogram-VUDS-SubsetsRowsInS.pdf}
\caption{Redshift distribution of the star-forming galaxy pairs 
(\textit{foreground} and \textit{background}) selected in this work.
The $\Delta$z = $z_{bg}$-$z_{fg}$ distribution is also included in the top-right 
corner.}
\label{Fig:PairsHistZ}%
\end{figure}  
%%%%%%%%%%%%%%%%%%%%%%%%%%%%%%%FIG_HISTO-RedShift%%%%%%%%%%%%%%%%%%%%%%%%%%%%%%

Stacking the \textit{fg} spectra provides the  absorptions at the foreground
galaxy's rest-frame. Figure \ref{Fig:PairsHistZ} shows the redshift
distributions of the \textit{foreground} (red) and
\textit{background} (blue) galaxy pairs and the \textit{fg-bg} pairs redshift
difference($\Delta z=z_{bg} -z_{fg}$) distribution on the top-right corner.
The spectroscopic sample includes 238 spectra within the redshift range 1.5
< z $\lesssim$ 4.43 ($\langle z \rangle$ = $2.6 \pm 0.41$) and is limited to
a 23$\arcsec$ maximum projected angular separation distance amongst galaxy
pairs, that correspond to 187.2 kpc at z=2.6. This sample is then split into
four different bins according to their angular projected separation:
$\theta\leqslant11\farcs8$, $11\farcs8<\theta\leqslant16\farcs5$,
$16\farcs5<\theta\leqslant20\arcsec$, and $20\arcsec<\theta\leqslant
23\arcsec$, identified as samples S1, S2, S3, and S4 and were
defined in such a way that each bin contains approximately the same number
of galaxies(see Table \ref{tbl:PairsStat}) providing a comparable S/N for
each composite spectra. These angular projected separation bins correspond
to projected physical distances (hereafter impact parameters) $b$:
$\leqslant$ 68.6\,kpc, 68.6\,kpc $< b \leqslant$113.5\,kpc, 113.5\,kpc $<
b \leqslant$ 146.2\,kpc, and 146.2\,kpc $< b \leqslant$ 172.8\,kpc for
samples S1, S2, S3, and S4. We note that the conversion from
angular separation to physical impact parameter varies $\pm3\%$ over the
full redshift range $(1.5 < z < 4.5)$ of the \textit{foreground} galaxy
sample, thus in the following impact parameter $b$ and projected angular
separation $\theta$ will be interchangeably used. Table \ref
{tbl:PairsStat} contains a summary of the statistical properties of the
galaxy pair samples. \\ 

To generate our \textit{fg} and \textit{bg} composite spectra we co-added
individual spectra as follows. For each pair of galaxies, the spectrum of
the \textit{fg} was:

\begin{enumerate}
    \item shifted to its own rest-frame;
    \item continuum-normalized using the full wavelength range\footnote
    {Full observed wavelength coverage: $3650 < \uplambda < 9350 \, [\AA]$ which
    at $z\sim$2.6 translates into a rest-frame wavelength coverage: $1013
    < \uplambda < 2597 \, [\AA]$}; 
    \item resampled\footnote{We have made use of {\sc{pysynphot}} \citep
    {Blagorodnova14} to resample our spectra.} to a common wavelength resolution
    $\Delta\uplambda$, defined by the maximum of the shifted wavelength
    resolution ($\Delta\uplambda_{shf\textunderscore fg}$) distribution of the
    galaxy sample used to generate the composite spectra
    (e.g. $\Delta\uplambda\sim2\AA$ for $fg$ composite spectra as shown in
    Figure \ref{Fig:Stk-Fg-Sep0-23});
    \item smoothed with a Gaussian kernel with width size $\Delta\uplambda$
    defined before.
\end{enumerate}

A similar approach was used to produce stacked spectra of the \textit
{bg} galaxies, each individual \textit{bg} galaxy spectrum was:

\begin{enumerate}
    \item shifted into the \textit {fg} galaxy's rest-frame using the same
     systemic redshift applied to their corresponding \textit{fg} galaxy
     spectrum;
    \item continuum normalized using the full wavelength range$^{1}$;
    \item resampled to a common wavelength resolution $\Delta\uplambda$,
     defined by the maximum of the shifted wavelength resolution
     ($\Delta\uplambda_{shf\textunderscore bg}$) distribution of the galaxy
     sample used to generate the composite spectra;
    \item smoothed with $\Delta\uplambda$ Gaussian kernel as defined before;
    \item masked the strong interstellar absorption features (see Fig 
    \ref{Fig:Stk-Fg-Sep0-23} and Table \ref{Tbl:Lines}) located at the
     redshift of each \textit{bg} rest-frame that can potentially
     contaminate the composite signal at the \textit{fg} rest-frame. 
\end{enumerate}

Finally, the \textit{foreground} and \textit{background} spectra were co-added
independently to produce composite stacked spectra; in all cases, for each
spectral bin, we calculated both the average and the median value, eventually
producing both an average and median co-added spectrum.

%%%%%%%%%%%%%%%%%%%%%%%%%%%%%%%%%%%%%%%%%%%%%%%%%%%%%%%%%%%%%%%%%%%%%%%%%%%%%%%
%%%%%%%%%%%%%%%%%%%%%%%%%%%%%%%%%%%ANALYSIS%%%%%%%%%%%%%%%%%%%%%%%%%%%%%%%%%%%%
%%%%%%%%%%%%%%%%%%%%%%%%%%%%%%%%%%%%%%%%%%%%%%%%%%%%%%%%%%%%%%%%%%%%%%%%%%%%%%%

%%%%%%%%%%%%%%%%%%%%%%%%%%%%%%%%%%%%%%%%%%%%%%%%%%%%%%%%%%%%%%%%%%%%%%%%%%%%%%%
%%%%%%%%%%%%%%%%%%%%%%%%%%%%%%%%%%%%RESULTS%%%%%%%%%%%%%%%%%%%%%%%%%%%%%%%%%%%%
%%%%%%%%%%%%%%%%%%%%%%%%%%%%%%%%%%%%%%%%%%%%%%%%%%%%%%%%%%%%%%%%%%%%%%%%%%%%%%%
\section{Results}\label{sec:Res}

Rest-frame UV spectra of star-forming galaxies at redshift $z\sim3$ are
commonly dominated by the emission of O and B stars; the CGM and/or IGM media
imprint absorption features on top of this UV continuum
\citep{Sargent80,Bergeron86,Bergeron91,Lanzetta95}. Composite spectra can
provide different l.o.s. of the average absorption strength produced by the
gas located in these media
\citep{Adelberger05a,Shapley03}. Figure~\ref{Fig:Stk-Fg-Sep0-23} shows the
median and average composite spectra by stacking the 238 \textit
{foreground} galaxies. The lower panel shows a histogram of the fraction of
galaxies contributing to each spectral bin of the co-added spectrum. In our
analysis, the separation zero ($b=0$) l.o.s. defines the interstellar medium
properties of the \textit{foreground} galaxies, while larger separations
($b>0$) l.o.s. define the properties of the CGM around the
\textit{foreground} galaxies at different separations.

While the l.o.s that we use to study the CGM cross through the whole CGM at a
given separation, the information about the ISM comes only from the
absorptions produced by the gas located in the half galaxy that is the
closest to the observer. For this reason, to compare the strength of the
absorptions at separation 0 (e.g. the ISM) and at separation $>0$(e.g. the
CGM), the former need to be corrected for this incompleteness.
Following \cite{Steidel10}, we apply the following factors: 1.45 for
low-ionization species(Si \,{\sc ii}, C \,{\sc ii}), 1.70 for Si \,{\sc iv}
and $\sim2$ for C \,{\sc iv}. The different l.o.s. probed by the \textit
{background} galaxy's light passing through the \textit{foreground} CGM help
us to trace neutral and ionized gas in H{\sc ii} star-forming regions and the
large-scale stellar winds produced by star-formation activity \citep
{Shapley03}. The main spectral features identified in our \textit
{fg} composite spectra (Figure \ref{Fig:Stk-Fg-Sep0-23}) in the
1100-2000$\AA$ range are summarized in Table \ref{Tbl:Lines} and include: H
{\sc i}(Ly$\upalpha$) emission/absorption (blue), low-ionization state metal
absorption lines (cyan), intermediate-ionization state metal absorption lines
(peach), high-ionization state metal absorption lines (magenta), interstellar
fine-structure emission line (green), absorption stellar photospheric lines
(gold), emission nebular lines (lime), and emission/absorption lines
associated with stellar winds (indigo)(\citealt
{Halliday08,LeFevre15,Shapley03}; OTELO 
\footnote{OSIRIS Tunable Emission Line Object survey \url{http://research.iac.es/proyecto/otelo/pages/data-tools/spectral-line-summary.php}}).

%%%%%%%%%%%%%%%%%%%%%%%%%%%%%%%%%%%TBL-LINES%%%%%%%%%%%%%%%%%%%%%%%%%%%%%%%%%%%
\begin{table}
\centering
\caption{\label{Tbl:Lines}
Main spectral features observed in our VUDS stacked 
spectra in the 1100-2000 $\AA$ rest-frame range.}
\begin{tabular}{ccc}
  \\
  \hline\hline
  \multicolumn{1}{c}{Spectral line} &
  \multicolumn{1}{c}{$\lambdaup$ [$\AA$]} &
  \multicolumn{1}{c}{Type} \\
  \hline\hline  
C  \,{\sc iii} & 1176 & Photospheric \\
Si \,{\sc ii}  & 1192 & LIS absorption \\
Si \,{\sc iii} & 1206 & IIS absorption \\
Ly$\upalpha$   & 1215.7 & H \,{\sc i} \\
N \,{\sc v}    & 1238/1242 & Photospheric \\ 
Si \,{\sc ii}  & 1260 & LIS absorption \\
Si \,{\sc ii*} & 1264.0 & Interstellar \\
O \,{\sc i}+Si \,{\sc ii}& 1303.2 & LIS absorption \\
Si \,{\sc ii*} & 1309.0 & Interstellar \\
C \,{\sc ii}   & 1334.5 & LIS absorption \\
O \,{\sc iv}   & 1343.0 & Photospheric \\
Si \,{\sc iv}  &  1393.8/1402.8  & HIS absorption\\
Si \,{\sc iii} & 1417 & Photospheric \\
S \,{\sc v}    & 1501.8 & Photospheric \\
Si \,{\sc ii}  & 1526.7 & LIS absorption \\
C \,{\sc iv}   &  1548.2/1550.8 & HIS absorption\\
Fe \,{\sc ii}  & 1608, & LIS absorption \\
Fe \,{\sc ii}  & 1610, & LIS absorption \\
He \,{\sc ii}  & 1640  & Stellar wind \\
O \,{\sc iii]} &  1660/1666 & Nebular \\
Al \,{\sc ii}  & 1670.7  & LIS absorption \\
Ni \,{\sc ii}  & 1709.6 & LIS absorption \\
N \,{\sc iv}   & 1718.5 & Stellar wind \\
Ni \,{\sc ii}  &  1741.5/1751.9 & LIS absorption \\
Si \,{\sc ii}  & 1808 & LIS absorption \\
Al \,{\sc iii} &  1854.7/1862.7 & IIS absorption \\
Si \,{\sc iii]}& 1889 & Nebular \\
C \,{\sc iii]} & 1908.7 & Nebular \\
\hline
\end{tabular}
\end{table}
%%%%%%%%%%%%%%%%%%%%%%%%%%%%%%%%%%%TBL-LINES%%%%%%%%%%%%%%%%%%%%%%%%%%%%%%%%%%%

%%%%%%%%%%%%%%%%%%%%%%%%%%%%%%%%%FIG-STACKS-FG%%%%%%%%%%%%%%%%%%%%%%%%%%%%%%%%%
   \begin{figure*}
   \centering
   \includegraphics[width=17cm]{PRP-ss-zf_B-3-44-ss-zf_F-3-44-ss-sep_as-0-23_Fg_stk_C_S-1100-2000.pdf}
   \caption{Top: Median (black) and average (red) composite spectra shifted to
    rest-frame of 238 \textit{foreground} galaxies with projected angular
    separations $<$\,23$\arcsec$($b<187.2$\,kpc at z=2.6). Bottom: The
    fraction of contributing individual spectra to the composite spectra per
    unit wavelength. Several spectral lines of interest are marked: H\,
    {\sc i} (Ly$\upalpha$) emission/absorption (blue), low-ionization state
    metal absorption lines (cyan), intermediate-ionization state metal
    absorption lines (peach), high-ionization state metal absorption lines
    (magenta), interstellar fine-structure emission line (green), absorption
    stellar photospheric lines (gold), emission nebular lines (lime), and
    emission/absorption lines associated with stellar winds (indigo).}
    \label{Fig:Stk-Fg-Sep0-23}%
    \end{figure*} 
%%%%%%%%%%%%%%%%%%%%%%%%%%%%%%%%%FIG-STACKS-FG%%%%%%%%%%%%%%%%%%%%%%%%%%%%%%%%%

%%%%%%%%%%%%%%%%%%%%%%%%%%%%%%%%%TBL-FG-BG-STAT%%%%%%%%%%%%%%%%%%%%%%%%%%%%%%%%
\begin{table*}
\centering
\caption{\label{tbl:PairsStat}
\textit{Foreground-background} (\textit{fg-bg}) galaxy pair statistics 
according to their projected angular separation ($\theta$).}
\begin{tabular}{cccccccccc}
  \\
  \hline\hline
  \multicolumn{1}{c}{ID} &
  \multicolumn{1}{c}{N} &
  \multicolumn{1}{c}{$\theta$} & 
  \multicolumn{1}{c}{$\langle \theta \rangle$} &
  \multicolumn{1}{c}{$b$ (kpc)} & 
  \multicolumn{1}{c}{$\langle b \rangle$ (kpc)} & 
  \multicolumn{1}{c}{$z_{\rm{fg}}$} &
  \multicolumn{1}{c}{$\langle \rm{z}_{\rm{fg}}\rangle$} &
  \multicolumn{1}{c}{$z_{\rm{bg}}$}  & 
  \multicolumn{1}{c}{$\langle \rm{z}_{\rm{bg}}\rangle$} \\
  \multicolumn{1}{c}{c1} & 
  \multicolumn{1}{c}{c2} &
  \multicolumn{1}{c}{c3} &
  \multicolumn{1}{c}{c4} &
  \multicolumn{1}{c}{c5} &
  \multicolumn{1}{c}{c6} &
  \multicolumn{1}{c}{c7} &
  \multicolumn{1}{c}{c8} &
  \multicolumn{1}{c}{c9} &
  \multicolumn{1}{c}{c10} \\
  \hline\hline  
ALL & 238 &  <23$\arcsec$             & 15$\farcs$7 $\pm$ 4$\farcs$9 &  13.8 - 187.2 & 125.0 $\pm$ 39.5 & 1.5-4.4 & 2.60 & 2.1-5.3 & 3.04 \\
S1  &  59 &  <11$\farcs$8             &  8$\farcs$7 $\pm$ 2$\farcs$0 &  13.8 -  97.8 & 68.6  $\pm$ 16.3 & 1.5-4.3 & 2.55 & 2.1-4.9 & 2.98 \\
S2  &  62 &   11$\farcs$8-16$\farcs$5 & 14$\farcs$4 $\pm$ 1$\farcs$4 &  92.6 - 135.9 & 113.5 $\pm$ 12.8 & 1.7-4.3 & 2.62 & 2.1-4.9 & 3.08 \\
S3  &  60 &   16$\farcs$5-20$\arcsec$ & 18$\farcs$4 $\pm$ 0$\farcs$9 & 127.1 - 164.6 & 146.2 $\pm$  8.8 & 1.9-3.6 & 2.66 & 2.3-4.5 & 3.11 \\
S4  &  57 &   20$\arcsec$-23$\arcsec$ & 21$\farcs$5 $\pm$ 0$\farcs$8 & 146.7 - 187.2 & 172.8 $\pm$  8.4 & 1.8-4.4 & 2.59 & 2.1-5.3 & 3.02 \\
  \hline\hline
\end{tabular}
\tablefoot{
Column 1: Sample ID; 
Column 2: number of galaxies per sample; 
Column 3: range of angular separation of the \textit{fg-bg} pairs;
Column 4: mean angular separation of the \textit{fg-bg} pairs within the
$\theta$ range;
Column 5: range of the impact parameter in kpc assuming $z_{med}$ within the
projected separation bin for conversion;
Column 6: mean impact parameter within the $\theta$ range in kpc;
Column 7 and 8: redshift ranges and mean redshift of the \textit
{foreground} galaxies sample within their corresponding $\theta$ range, and
Columns 9 and 10: redshift ranges and mean redshift of the \textit
{background} galaxies sample within their corresponding $\theta$ range.}
\end{table*}
%%%%%%%%%%%%%%%%%%%%%%%%%%%%%%%%%TBL-FG-BG-STAT%%%%%%%%%%%%%%%%%%%%%%%%%%%%%%%%

In the 900--1900\,$\AA$ range, we have explored the following absorption lines:
Ly$\upalpha$ ($\uplambda$ 1215.7 $\AA$), 
O\,{\sc i}+Si\,{\sc ii} ($\uplambda\uplambda$ 1303.2 $\AA$), 
C\,{\sc ii} ($\uplambda$ 1334.5 $\AA$), 
Si\,{\sc iv} ($\uplambda\uplambda$ 1393-1402 $\AA$), 
Si\,{\sc ii} ($\uplambda$ 1526.7 $\AA$), 
C\,{\sc iv} ($\uplambda\uplambda $ 1548.2-1550.8 $\AA$), 
Fe\,{\sc ii} ($\uplambda$ 1608.5 $\AA$), 
Al\,{\sc ii} ($\uplambda$ 1670.8 $\AA$) and 
Al\,{\sc iii} ($\uplambda$ 1862.8 $\AA$).
First we present their equivalent widths as a function of the angular
separation between galaxy pairs ($\theta$) (see Sect. \ref{sec:EW-Rad}). We
then examine the correlations between the strength of Ly$\upalpha$, LIS (C\,
{\sc ii}, Si\,{\sc ii}) and HIS (C\,{\sc iv}, Si\,{\sc iv}) lines as a
function of different galaxy properties, including stellar mass, star
formation rate (see Sect.~\ref{sec:EW-SFR}), effective radius ($r_{\rm
{eff}}$) and the azimuthal angle ($\phi$) (see Sect.~\ref{sec:EW-Mrph}). We
also explore the C\,{\sc ii}/C\,{\sc iv} line ratio (see Sect.~\ref
{sec:LR}) as a function of the projected angular separation, stellar mass,
and star formation rate.

%%%%%%%%%%%%%%%%%%%%%%%%%%%%%%%%%%%%%%%%%%%%%%%%%%%%%%%%%%%%%%%%%%%%%%%%%%%%%%%
%%%%%%%%%%%%%%%%%%%%%%%%%%%%RES-RADIAL DEPENDENCE%%%%%%%%%%%%%%%%%%%%%%%%%%%%%%
%%%%%%%%%%%%%%%%%%%%%%%%%%%%%%%%%%%%%%%%%%%%%%%%%%%%%%%%%%%%%%%%%%%%%%%%%%%%%%%
\subsection{Radial dependence}\label{sec:EW-Rad}

Here we present absorption lines produced by gas located in the CGM of
star-forming galaxies at $z\sim2.6$. Figure~\ref{Fig:Lns-Bg-Sep-4} shows the
average absorption spectra obtained after stacking the \textit
{background} galaxy spectra at the redshift of the
\textit{foreground} galaxy as a function of their projected angular
separation: $<11\farcs8$, $11\farcs8$--$16\farcs5$, $16\farcs5$--$20\arcsec$
and $20\arcsec$--$23\arcsec$. We note that as a consequence of VUDS design
and selection criteria, the number of galaxies with small projected angular
separations($<6\arcsec$) is scarce. The strengths of the line absorptions
were measured by fitting a single Gaussian profile to the stacked spectra.
Equivalent widths ($W_{0}$) of each line profile are obtained by integrating
the Gaussian fit of each stacked spectrum around the central wavelength of
the Gaussian fit to the line using an integration window within the
$\pm5\sigma$ range.

All equivalent width measurements are given in the rest-frame and we use
positive (negative) equivalent widths to indicate absorption (emission). The
errors on the equivalent width measurements are determined using a bootstrap
approach. For each set of galaxy \textit{background} spectra, a thousand
co-added spectra are generated from random selections, with replacements, from
that same sample, preserving the number of evaluated sources. For each of
these co-adds, the line absorption equivalent widths are estimated. The 
16th/84th percentiles of the distribution of equivalent widths are taken as
error intervals for the original measurement. We note that owing to the nature 
of star-forming galaxies, that show a diversity of spectral features 
(e.g., ISM absorption lines, photospheric stellar lines/nebular lines, and
Ly$\upalpha$ emission), the absorption lines in our composite spectra show
both an emission and absorption component. The
emission components can be produced by the same transition responsible 
for producing the absorption line (e.g. Ly$\upalpha$), 
produced by nearby fine-structure transitions or 
in some cases they can be completely absent.

%%%%%%%%%%%%%%%%%%%%%%%%%%%%%%%FIG-LINE-DETECTIONS%%%%%%%%%%%%%%%%%%%%%%%%%%%%%
\begin{figure*}
\centering
\includegraphics[width=17cm]{P_Bg_ALL_0_0-40-PRP-ss-zf_B-3-44-ss-zf_F-3-44-ss-sep_as-0-23-ss_sep_as-20-23-0-11.8-20-23-ALL-med-c-smt-G-1200-1900-SpltEX.pdf}
  \caption{Median absorption lines (thin lines) detected from 
    our 238 \textit{foreground-background} galaxy pairs, including
    all \textit{foreground} galaxies (red) and \textit{background} galaxies
    split by their projected angular separations:
    <11$\farcs$8 (blue), 
    11$\farcs$8-16$\farcs$5 (green), 
    16$\farcs$5-20$\arcsec$ (cyan), and 
    20$\arcsec$-23$\arcsec$ (magenta); 
    bold lines indicate
    the Gaussian profile fit to measure the 
    strength of the line absorptions (see Sect. \ref{sec:EW-Rad}).}
\label{Fig:Lns-Bg-Sep-4}
\end{figure*}
%%%%%%%%%%%%%%%%%%%%%%%%%%%%%%%FIG-LINE-DETECTIONS%%%%%%%%%%%%%%%%%%%%%%%%%%%%%

Figure \ref{Fig:EW-Pfl-Bg-sep-4} shows the radial profiles of the absorption
rest-frame equivalent width obtained from the average (red) and median
(blue) stacked absorption specrta presented in Figure~\ref
{Fig:Lns-Bg-Sep-4}. Median composite spectra are expected to be free of
artefacts and contamination such as sky residuals, unexpected
absorption/emission features in the \textit{background} galaxy spectra. On
the other hand, the average composite spectra could be affected by strong
absorption lines coming from individual l.o.s. Median and average composite
spectra are in general concordant, tracing similar trends. We also tested
other combination techniques, including straight average,
average-sigma-clipping and average-continuum-weighted, that produce very
similar composite spectra. Table \ref{tbl:EW-SEP} contains a summary of the
measured equivalent widths as a function of the projected angular separation,
including the significance (in terms of S/N) of the detections. Noise
estimations were computed with the 
{\sc DER$\_$SNR}\footnote{\url{http://www.stecf.org/software/ASTROsoft/DER_SNR/}} 
algorithm  \citep{Stoehr08} on continuum band-pass adjacent to each central
absorption line. We limit our analyses to detections with S/N\,$\geq$\,3,
however, upper limits with $S/N\,<\,3$ are also included and highlighted
using solid black symbols and arrows in Figure \ref{Fig:EW-Pfl-Bg-sep-4}, as
these detections are useful to outline the possible correlations/trends of
the profiles. We note that some precautions should be taken when interpreting
the absorption strengths of these spectral features as they could not reflect
the true conditions of the cold and hot gas CGM components. For instance, the
balances amongst singly ionized species of some elements (e.g. Al, Cl) can be
altered by dielectronic recombination \citep{Black73,Watson73,Jura74} or
ion-molecule reactions(charge exchange reactions of ionized species with
neutral hydrogen and helium \citealt{Steigman75b}). In fact, while the
abundances relative to neutral hydrogen of some elements (e.g. O, N, P, Cl,
Si, Cr, Mn, Fe, Ni) can be affected by factors of 2-10, Al is depleted  by a
factor of $\sim$ 100 \citep{York79}, making Al\,{\sc ii} a biased tracer of
the neutral gas phase. In addition, Fe\,{\sc ii} is contaminated by Fe\,
{\sc iv} ($\uplambda$ 1610 $\AA$) and Fe\,{\sc ii} ($\uplambda$ 1611.5 $\AA$)
lines \citep{Judge92,Shull83,Pickering02}, making it difficult to determine
accurately its absorption strength. Subsequently, double ionized species
(e.g. Si\,{\sc iii}, Al\,{\sc iii}) are intermediate-ionization state
(IIS) tracers of moderately photoionized warm gas sensitive to both diffuse
ionized gas (traced by HIS) as well as denser, partly neutral gas. Depending
on the species they can be correlated with cold (e.g. Si\,{\sc iii} \citealt
{Richter16}) or hot gas components (e.g. Al\,{\sc iii} \citealt
{Vladilo01}) and they can be used to infer the relative mix of neutral and
ionized material 
\citep{Howk99b}.

%%%%%%%%%%%%%%%%%%%%%%%%%%%%%%%%%FIG-EW-vs-SEP%%%%%%%%%%%%%%%%%%%%%%%%%%%%%%%%%
   \begin{figure*}
   \centering
   \includegraphics[height=21cm]{P_Bg_ALL_0_0-40-PRP-ss-zf_B-3-44-ss-zf_F-3-44-ss-sep_as-0-23-ss-0-11.8-0-23-gauss-1200-1900-med-avg-c-smt-lmfit-EW-Radial-ALL-lmfit_m-lmfit_e-GEX.pdf}
   \caption{
   Rest equivalent width ($W_{0}$) as a function of the impact 
   parameter ($b$) obtained from the line profiles in Fig \ref{Fig:Lns-Bg-Sep-4},
   corresponding to the \textit{foreground} composite spectra (0 kpc) and the
   \textit{background} composite spectra at $\langle b \rangle$
   68.6 kpc (8$\farcs$7), 
   113.5 kpc (14$\farcs$4), 
   146.2 kpc (18$\farcs$4) and 
   172.8 kpc (21$\farcs$5). 
   Average and median $W_{0}$ are shown in solid-red and open-blue symbols,
   solid black symbols correspond to to upper limits with SNR
   <3. Reported values from the literature including low-redshift values
   (open-gray symbols):
   \citealt{Bordoloi18} (stars), 
   \citealt{Borthakur15} (pentagons), 
   \citealt{Liang14} (crosses), 
   \citealt{Prochaska11}  (diamonds), 
   \citealt{Werk13} (pluses), and  
   high-redshift values \citealt{Steidel10} black triangles. 
   Error bars correspond to 1$\sigma$ confidence intervals for average
   (red filled circles) or median (blue unfilled circles) values based on a
   bootstrap analysis. Additionally, the results from the Kendall-Tau
   correlation test: the correlation coefficients $r_{\tau}$ and their
   corresponding $p$-value (probability of no correlation) are also
   included.}
   \label{Fig:EW-Pfl-Bg-sep-4}%
   \end{figure*}   
%%%%%%%%%%%%%%%%%%%%%%%%%%%%%%%%%FIG-EW-vs-SEP%%%%%%%%%%%%%%%%%%%%%%%%%%%%%%%%%

We note, that the composite spectra considering \textit{foreground}
(down-the-barrel) galaxies from our $fg-bg$ galaxy pairs split by separation,
do not show any significant dependency of the $W_{0}$ of absorption lines
with impact parameter $b$, that show an average variance of $\leq15\%$
amongst the absorption lines included in \ref{Fig:EW-Pfl-Bg-sep-4}. This is
in agreement with the $< 10\%$ variance that \cite{Steidel10} reported for
their brightest absorption lines. Considering the \textit
{background} galaxies from our $fg-bg$ galaxy pairs split by separation, our
radial profiles show a general negative gradient as reported by previous
works. As the impact parameter increases, the strength of the detected
absorption feature decreases. This is the case for all the spectral lines
considered in our analysis. Recently, \cite{Kacprzak21} compiled measurements
of low-ionization state absorption lines(O\,{\sc i}, C\,{\sc ii}, N\,
{\sc ii}, Si\,{\sc ii}),  and high-ionization state absorption lines(C\,
{\sc iii}, N\,{\sc iii}, Si\,{\sc iii}, C\,{\sc iv}, Si\,{\sc iv}, N{\sc v},
and O{\sc vi}) in the CGM of low-redshift($z \lesssim 0.3$) galaxies with
$9\leq \rm{log[M}_{\bigstar}/\rm{M}_{\odot}]\leq11$, using quasars as
background spotlights. Measurements at low redshift
\citep{Bordoloi18,Borthakur15,Johnson17,Liang14,Prochaska11,Werk13} and
high-redshift ($z \sim 2.3$) galaxies \citep{Steidel10} are also overplotted
in Figure~\ref{Fig:EW-Pfl-Bg-sep-4}. 

Although at low redshift these lines can be detected at distances up to
300\,kpc, at high redshift using galaxy-galaxy pairs, 
\cite{Steidel10} reported detections that are limited to 100 kpc; including 42
additional bright quasars-galaxy pairs that provide access to the
low-density gas in the CGM, they extended their Ly$\upalpha$ detections up
to 280\,kpc. More recently, using galaxy images at $\langle
z \rangle \sim$\,2.4,
\cite{Chen20} measured the Ly$\upalpha$ excess relative to the background
intergalactic medium, probing the CGM gas up to impact parameters of
2000\,kpc. Here, for star-forming galaxies at a mean redshift $\langle
z \rangle$ = 2.6, we report detections at distances up to 146\,kpc and
172\,kpc  for HIS and LIS/Ly$\upalpha$. These results include
Al\,{\sc iii}, which is detected for the first time in the CGM of
high-redshift star-forming galaxies up to distances of $\sim$150\,kpc.
Low-ionization state lines (C\,{\sc ii}, Si\,{\sc ii}, Al\,{\sc ii}) show
less steep radial profiles compared with high ionization lines. The LIS
radial profile shows an abrupt decay at smaller radii. From separation zero
to $\langle b \rangle\sim 68$\,kpc, the strengths of C\,{\sc ii}, Si\,
{\sc ii}, and Al\,{\sc ii} are reduced by a factor of 1.7, 2.5, and 3.2,
while Si\,{\sc iv} and C\,{\sc iv} are reduced by a factor of 6.8 and 4.5.

\cite{Steidel10} reported that at a separation of $\langle b \rangle$ =
103\,kpc, lines can be hardly detected in their stacked spectra,
low-ionization state metal absorption lines cannot be detected and C\,
{\sc iv} is detected with marginal significance. Here we are able to detect
LIS (C\,{\sc ii} and Si\,{\sc ii}) and HIS (Si\,{\sc iv}, C\,{\sc iv}) both
with S/N$\geq3$ up to $\langle b \rangle=$\,172\,kpc and 146\,kpc
separations. We note that beyond this separation
($\theta>23\arcsec$=172\,kpc) we are unable to detect any significant signal
from the \textit{background} composite spectra. This might be caused by
differences in the S/N and resolution of the individual spectra used in the
two analyses. The spectral line features considered here are more difficult
to detect in our low-resolution spectra compared to a high-resolution
spectra at the same S/N level. Higher-resolution spectra could allow to
resolve spectral lines that are blended (e.g. O\,{\sc i}-Si\,{\sc ii}:
$\uplambda\uplambda$ 1303-1307 $\AA$, Si\,{\sc iv}: $\uplambda\uplambda$
1393-1402 $\AA$), or that are affected by close
photospheric/interstellar/nebular spectral lines(e.g. Fe\,{\sc ii}-Fe\,
{\sc iv}:$\uplambda\uplambda$1608-1610, Si\,{\sc ii}-Si\,{\sc ii}*:
$\uplambda\uplambda$ 1260-1264), disentangling the spectral features and
thus provide a more accurate $W_{0}$ measurements. Another possibility to
explain these discrepancies, is that they might be the result of differences
in the SNR of both parent dataset samples. We know that both parent samples
adopted opposite observational strategies. On one hand, 
\cite{Steidel04} deliberately kept short the total exposure times on their
observations to maximize the number of galaxies for which redshifts could be
measured. This led to a dataset with spectral quality (S/N) that varies
considerably amongst their spectra \citep{Steidel04,Steidel10}, and that
could be affecting their redshift determinations. On the other, VUDS provide
a homogeneous dataset with integration times of $\simeq $14h per target,
allowing to reach a S/N on the continuum at 8500 $\AA$ of S/N=5 \citep
{LeFevre15}. Moreover, in this work, we have considered galaxies with
reliability flags ($z_{f}$ = 3, 4) with 95-100\% probability of being
correct. However, it is not clear if similar constraints on their redshift
determinations were adopted by \cite{Steidel10} to define their
galaxy-galaxy pair sample. This is crucial, as composite spectra can be
affected by spectral offsets in the individual spectra. LIS and HIS spectral
absorptions could be washed out or artificially boosted by considering
individual spectra with low SNR and low-reliability redshift determinations
to generate composite spectra. In fact, the spectra considered to generate
our stacked spectra may show an average SNR larger than the average SNR of
the spectra considered by \cite{Steidel10}. If similar separations $\langle
b\rangle $ are considered, Steidel's P3 sample at $\langle b\rangle =103$
kpc ($\langle \theta \rangle$ = 12$\farcs$5) and this work S2 sample
$b=113.5$ kpc($\langle \theta \rangle$ = 14$\farcs$4), Steidel's P3 sample
considers $\sim$ 5 times the number of objects (N=306) considered in our S2
sample(N=62). This would imply that our spectra have an average higher
individual SNR compared to the average SNR of the dataset used by \cite
{Steidel10}, allowing us to detect similar/weaker spectral absorptions at
larger separations.

To assess the correlation between $W_{0}$ and $b$, we implemented a
Kendall-Tau correlation test. The correlation coefficients $r_{\tau}$ and
their corresponding $p$-value(the probability of no correlation) are provided
in the same figure.  In all cases we find a robust anticorrelation of the
strength of the absorption line as a function of the impact parameter: in
particular, Ly$\upalpha$, C\,{\sc ii}, C\,{\sc iv}, and Fe\,{\sc ii} all show
a strong correlation($p<0.05$), while we find that O\,{\sc i}, Si\,{\sc iv},
Si\,{\sc ii}, Al\,{\sc ii}, and Al\,{\sc iii} present a flatter radial
profile with that results in a lower significance of the anti-correlation
($p\sim0.1$). To assess the scale of the relationship between $W_{0}$ and
$b$, we compute the slopes of the radial profiles shown in Fig \ref
{Fig:EW-Pfl-Bg-sep-4}. We find that Ly$\alphaup$ show a slope of
-1.20. $\pm$ 0.01, LIS: O\,{\sc i}+Si\,{\sc ii}, C\,{\sc ii}, Si\,{\sc ii},
Fe\,{\sc ii}, and Al\,{\sc ii} absorption lines show slopes -1.34 $\pm$
0.01, -1.31  $\pm$  0.01, -1.53  $\pm$  0.02, -1.50  $\pm$  0.07, and -1.44
$\pm$ 0.05,  Al\,{\sc iii} show a slope of  -1.38 $\pm$ 0.06
and HIS: Si\,{\sc iv} and C\,{\sc iv}  absorption lines show, -1.10  $\pm$
0.04 and -1.24 $\pm$  0.01. If we focus on low and high
ionization lines of the same species, and considered Si\,{\sc ii}--Si\,
{\sc iv} and C\,{\sc ii}--C\,{\sc iv} absorption line pairs, we find that
the slopes of these absorption lines are different at $\geq5\sigma$ level.
Compared with \cite{Steidel10} radial profiles, our results show slopes that
are different at a 5$\sigma$ level for Si\,{\sc ii} and 3$\sigma$ level for
C\,{\sc ii}  and Al\,{\sc ii} absorption lines. These differences suggest
that within the CGM, cold and dense gas is more spatially extended in
galaxies at $z\sim2.6$ compared with galaxies at $z\sim2.3$ and lower
redshifts, as probed by Si\,{\sc ii} C\,{\sc ii}, in agreement with the
expected higher covering factors of neutral gas at higher redshifts \citep
{Reddy16,Du18,Sanders21}. Compared with low-redshift CGM studies
(see Figure~\ref{Fig:EW-Pfl-Bg-sep-4}), our Ly$\upalpha$, LIS and HIS
rest-frame equivalent width radial profiles are at the upper envelope of the
equivalent width measurements at lower redshifts, suggesting a potential
redshift evolution for the CGM gas content that produces these absorptions.

%%%%%%%%%%%%%%%%%%%%%%%%%%%%%%%%%TBL-EW-vs-SEP%%%%%%%%%%%%%%%%%%%%%%%%%%%%%%%%%
\begin{table*}
\centering
\caption{\label{tbl:EW-SEP} Median absorption line strengths measured ($W_
 {0} \, \AA$) in stacked spectra as a function of the average impact
 parameter ($\langle b \rangle$).}
\begin{tabular}{ccccccccccccccccccccccccccccccccccccccccc}
  \\
  \hline\hline
  \multicolumn{1}{c}{$\langle b \rangle$ (kpc)}  &
  \multicolumn{1}{c}{$W_{0}$ [$\AA$]}&
  \multicolumn{1}{c}{S/N} &
  \multicolumn{1}{c}{$W_{0}$ [$\AA$]}&
  \multicolumn{1}{c}{S/N} &
  \multicolumn{1}{c}{$W_{0}$ [$\AA$]}&
  \multicolumn{1}{c}{S/N} &
  \multicolumn{1}{c}{$W_{0}$ [$\AA$]}&
  \multicolumn{1}{c}{S/N} &
  \multicolumn{1}{c}{$W_{0}$ [$\AA$]}&
  \multicolumn{1}{c}{S/N} \\
  \multicolumn{1}{c}{c1} &
  \multicolumn{1}{c}{c2} &
  \multicolumn{1}{c}{c3} &
  \multicolumn{1}{c}{c4} &
  \multicolumn{1}{c}{c5} &
  \multicolumn{1}{c}{c6} &
  \multicolumn{1}{c}{c7} &
  \multicolumn{1}{c}{c8} &
  \multicolumn{1}{c}{c9} &
  \multicolumn{1}{c}{c10} &
  \multicolumn{1}{c}{c11} \\
  \hline\hline
  &
  \multicolumn{2}{c}{Ly$\upalpha$ 1215.7} &
  \multicolumn{2}{c}{O\,{\sc i}-Si\,{\sc iii} 1303.2} &
  \multicolumn{2}{c}{C\,{\sc ii} 1334.5} &
  \multicolumn{2}{c}{Si\,{\sc iv} 1393.8} &
  \multicolumn{2}{c}{Si\,{\sc iv} 1402.8} \\
  \hline\hline  
  0.0   & 3.54 $\pm$ 0.02 & 58 & 2.61 $\pm$ 0.15 & 17  & 2.31 $\pm$ 0.08 & 20 & 1.91 $\pm$ 0.01 & 18  & 1.40 $\pm$ 0.05 & 67\\
  68.6  & 1.71 $\pm$ 0.13 & 14 & 0.97 $\pm$ 0.10 &  9  & 1.37 $\pm$ 0.02 &  4 & 0.28 $\pm$ 0.06 &  4  & 0.32 $\pm$ 0.06 & 15\\
  113.5 & 0.75 $\pm$ 0.05 &  9 & 0.59 $\pm$ 0.07 &  8  & 0.54 $\pm$ 0.01 &  3 & 0.09 $\pm$ 0.04 &  2  &      <0.10     & <2\\
  146.2 & 1.08 $\pm$ 0.13 & 13 &      <0.26      & <2  & 0.30 $\pm$ 0.09 &  2 & 0.13 $\pm$ 0.19 &  3  & 0.27 $\pm$ 0.31 & 5 \\
  172.8 & 0.46 $\pm$ 0.14 &  5 &      <0.39      & <2  & 0.38 $\pm$ 0.10 &  3 &      <0.09      & <2  &      <0.06     & <2\\
  \hline\hline
  &
  \multicolumn{2}{c}{Si\,{\sc ii} 1526.7} &
  \multicolumn{2}{c}{C\,{\sc iv}  1548.2} &
  \multicolumn{2}{c}{Fe\,{\sc ii} 1608.5} &
  \multicolumn{2}{c}{Al\,{\sc ii}  1670.8} &
  \multicolumn{2}{c}{Al\,{\sc iii} 1862.8} \\  
  \hline\hline
  0.0   & 0.94 $\pm$ 0.02 & 18 & 2.87 $\pm$ 0.11 & 20  & 0.99 $\pm$ 0.01 &  6  & 1.13 $\pm$ 0.03 & 10  & 1.25 $\pm$ 0.01 &  8   \\
  68.6  & 0.35 $\pm$ 0.03 & 3  & 0.70 $\pm$ 0.04 & 13  & 0.18 $\pm$ 0.03 &  3  & 0.35 $\pm$ 0.03 &  4  & 0.31 $\pm$ 0.11 &  7   \\
  113.5 & 0.24 $\pm$ 0.09 & 2  & 0.27 $\pm$ 0.04 & 2   &       <0.03     &  <2 & 0.18 $\pm$ 0.03 &  3  &       <0.06     & <2  \\
  146.2 & 0.36 $\pm$ 0.05 & 4  & 0.24 $\pm$ 0.02 & 3   &       ----      & --- &       ----      & --- & 0.40 $\pm$ 0.02 &  3   \\
  172.8 & 0.39 $\pm$ 0.04 & 4  &      <0.05      & <2  &       ----      & --- &       ----      & --- &       <0.34     & <2  \\
\hline\hline 
\end{tabular}
\tablefoot{
\textit{Foreground-background (fg-bg)} galaxy pairs split by their average impact parameters 
($\langle b \rangle$):
68.6\,kpc (N=60), 
113.5\,kpc (N=62), 
146.2\,kpc (N=60), and
172.8\,kpc (N=57). 
The errors on the equivalent width ($W_{0}$) measurements 
correspond to 1$\sigma$ confidence intervals 
based on a Bootstrap analysis (see Sect. \ref{sec:EW-Rad}).
}
\end{table*}
%%%%%%%%%%%%%%%%%%%%%%%%%%%%%%%%%TBL-EW-vs-SEP%%%%%%%%%%%%%%%%%%%%%%%%%%%%%%%%%

%%%%%%%%%%%%%%%%%%%%%%%%%%%%%%%%%%%%%%%%%%%%%%%%%%%%%%%%%%%%%%%%%%%%%%%%%%%%%%%
%%%%%%%%%%%%%%%%%%%%%%%%%%%%RES-RADIAL DEPENDENCE%%%%%%%%%%%%%%%%%%%%%%%%%%%%%%
%%%%%%%%%%%%%%%%%%%%%%%%%%%%%%%%%%%%%%%%%%%%%%%%%%%%%%%%%%%%%%%%%%%%%%%%%%%%%%%


%%%%%%%%%%%%%%%%%%%%%%%%%%%%%%%%%%%%%%%%%%%%%%%%%%%%%%%%%%%%%%%%%%%%%%%%%%%%%%%
%%%%%%%%%%%%%%%%%%%%%%%%%%%RES SFR & STELLAR MASS%%%%%%%%%%%%%%%%%%%%%%%%%%%%%%
%%%%%%%%%%%%%%%%%%%%%%%%%%%%%%%%%%%%%%%%%%%%%%%%%%%%%%%%%%%%%%%%%%%%%%%%%%%%%%%
\subsection{Star formation and stellar mass Dependence}\label{sec:EW-SFR}

%%%%%%%%%%%%%%%%%%%%%%%%%%%%%%%%%FIG-MRPH-ALL%%%%%%%%%%%%%%%%%%%%%%%%%%%%%%%%%%
%%%%%%%%%%%%%%%%%%%%%%%%%%%%%%%%%%EW-vs-REF%%%%%%%%%%%%%%%%%%%%%%%%%%%%%%%%%%%%
\begin{figure*}
\begin{tabular}{cc}
  \includegraphics[width=85mm]{P_Bg_ALL_0_0-40-PRP-ss-zf_B-3-44-ss-zf_F-3-44-ss-sep_as-0-23-ss-8-9.5-10.02-14-gauss-1200-1900-med-avg-c-smt-lmfit-EW-Mass-ALL-lmfit_m-lmfit_e-GEX.pdf} &
  \includegraphics[width=85mm]{P_Bg_ALL_0_0-40-PRP-ss-zf_B-3-44-ss-zf_F-3-44-ss-sep_as-0-23-ss-0-1.1-1.66-3.0-gauss-1200-1900-med-avg-c-smt-lmfit-EW-SFR-ALL-lmfit_m-lmfit_e-GEX.pdf} \\
%(a) first & (b) second \\[6pt]
 \includegraphics[width=85mm]{P_Bg_ALL_0_0-40_PRP_MRP-ss-zf_B-3-44-ss-zf_F-3-44-ss-sep_as-0-23-ss-0-1-1-2.3-gauss-1200-1900-med-avg-c-smt-lmfit-EW-r_eff-ALL-lmfit_m-lmfit_e-GEX.pdf} &
 \includegraphics[width=85mm]{P_Bg_ALL_0_0-40_PRP_MRP-ss-zf_B-3-44-ss-zf_F-3-44-ss-sep_as-0-23-ss-0-30-60-90-gauss-1200-1900-med-avg-c-smt-lmfit-EW-PHI-ALL-lmfit_m-lmfit_e-GEX.pdf} \\
%(c) third & (d) fourth \\[6pt]
\end{tabular}
\caption{Rest equivalent width ($W_{0}$) of the absorption lines as a function
 of the \textit{foreground} galaxy's stellar mass (log[M$_\bigstar$/M$_
 {\odot}$]), SFR (log[M$_{\odot}$yr$^{-1}$]), effective radius ($r_{\rm
 {eff}}$) and azimuthal angle($\phi$). The $W_{0}$ measurements come from
 composite spectra considering galaxy pairs at all projected
 distances ($b < 23 \arcsec$) and split by the corresponding galaxy property.
 Stellar mass and SFR $W_{0}$ were obtained from composite spectra
 considering 238 \textit{bg} galaxies, while $r_{\rm{eff}}$ and $\phi$ come
 from composite spectra considering 97 \textit{bg} galaxies(see Sect. \ref
 {sec:Data} and Table \ref{Tbl:Samples}). Average and median $W_{0}$ are
 shown in solid-red and open-blue symbols, solid black symbols
 correspond to upper limits with SNR <3. Error bars correspond to 1$\sigma$
 confidence intervals for average or median values based on a Bootstrap
 analysis. The panels include the results from the Kendall-Tau correlation
 test: the correlation coefficients $r_{\tau}$ and their corresponding
 $p$-value (the probability of no correlation). The panels of absorption
 lines with detections in only two subsamples also show the results from
 Student's t-test:  the difference between a pair of mean values given by the
 $t$ coefficient and their corresponding $p$-value(the probability of
 significant difference amongst means).}
\label{Fig:EW-Pfl-Bg-PRP-ALL}
\end{figure*}
%%%%%%%%%%%%%%%%%%%%%%%%%%%%%%%%%%EW-vs-REF%%%%%%%%%%%%%%%%%%%%%%%%%%%%%%%%%%%%
%%%%%%%%%%%%%%%%%%%%%%%%%%%%%%%%%FIG-MRPH-ALL%%%%%%%%%%%%%%%%%%%%%%%%%%%%%%%%%%

To explore the dependence of low- and high-ionization state absorptions with
physical and morphological properties (i.e. star formation rate, stellar
mass, galaxy effective radius and azimuthal angle) produced in the CGM  of
our high-redshift star-forming galaxy sample, we considered galaxy pairs at
all projected distances ($b < 23 \arcsec$) split by the corresponding galaxy
property. A summary of the statistical properties of the galaxy pair
subsamples divided by these properties can be found in the Appendix \ref
{sec:AppendixTbl}. We note that Fe\,{\sc ii}, Al\,{\sc iii}, and Al\,{\sc ii}
lines were excluded from our following analyses (see sec \ref
{sec:EW-Rad}) Figure \ref{Fig:EW-Pfl-Bg-PRP-ALL} (upper panels) shows the
equivalent width of the absorption features as a function of stellar mass and
SFR for LIS (C\,{\sc ii}, Si\,{\sc ii}), HIS (Si\,{\sc iv}, C\,{\sc iv}) and
Ly$\upalpha$. We find that galaxies with high stellar masses (log[M$_
{\bigstar}$/M$_{\odot}$]>10.30) and high star formation rates (log[SFR/(M$_
{\odot}$yr$^{-1}$)]>1.93) show C\,{\sc iv} (HIS) metal absorption with larger
equivalent widths. On the other hand, galaxies with low stellar masses (log
[M$_{\bigstar}$/M$_{\odot}$]<9.26) and low SFRs (log[SFR/(M$_{\odot}$yr$^
{-1})]<0.9$) show C\,{\sc ii} (LIS) metal absorptions with stronger
equivalent widths. Nevertheless,  Si\,{\sc ii} and  Si\,{\sc iv} do not show
a similar trend. This might be caused by the fact that we do not have Si\,
{\sc ii} and  Si\,{\sc iv} detections with S/N>3 in all of our SFR and
stellar mass bins.

To quantify the robustness of the trends highlighted in Figure~\ref
{Fig:EW-Pfl-Bg-PRP-ALL}, we ran a Kendall-Tau rank test for the absorption
lines that are detected in at least 3 bins. We find that SFR and stellar mass
are anti-correlated with C\,{\sc ii}, but they correlate positively with C\,
{\sc iv}. This would imply that C\,{\sc ii} is located mostly in
galaxies with low stellar mass(log[M$_{\bigstar}$/M$_{\odot}$]<9.26) and low
SFR (log[SFR/(M$_{\odot}$yr$^{-1})]<0.9$) while C\,{\sc iv} is detected in
galaxies with high stellar mass (log[M$_{\star}$/M$_{\odot}$]>10.2) and high
SFR (log[SFR/(M$_{\odot}$yr$^{-1})]>1.5$). In fact, the slopes of the $W_
{0}$ relationships with stellar mass and SFR for C\,{\sc ii} and C\,{\sc iv}
are 0.79 $\pm$ 0.13 , 0.61 $\pm$ 0.05 and 0.73 $\pm$ 0.08 , 0.55 $\pm$ 0.04,
that are significantly different at $\geq5\sigma$ level. We do not find any
robust correlation between SFR and stellar mass with Si\,{\sc ii}, Si\,
{\sc iv} Fei\,{\sc ii}, Al\,{\sc ii}, Al\,{\sc iii} or Ly$_
{\upalpha}$, yielding a probability of no correlation above 60\% in all
cases. We also explored the combined effect of impact parameter with stellar
mass and SFR; however, no clear conclusion was reached due to low-number
statistics resulting in composite spectra with low S/N line detection.
%%%%%%%%%%%%%%%%%%%%%%%%%%%%%%%%%%%%%%%%%%%%%%%%%%%%%%%%%%%%%%%%%%%%%%%%%%%%%%%
%%%%%%%%%%%%%%%%%%%%%%%%%%%RES SFR & STELLAR MASS%%%%%%%%%%%%%%%%%%%%%%%%%%%%%%
%%%%%%%%%%%%%%%%%%%%%%%%%%%%%%%%%%%%%%%%%%%%%%%%%%%%%%%%%%%%%%%%%%%%%%%%%%%%%%%

%%%%%%%%%%%%%%%%%%%%%%%%%%%%%%%%%%%%%%%%%%%%%%%%%%%%%%%%%%%%%%%%%%%%%%%%%%%%%%%
%%%%%%%%%%%%%%%%%%%%%%%%%%%%%%%%%RES-MORPHO%%%%%%%%%%%%%%%%%%%%%%%%%%%%%%%%%%%%
%%%%%%%%%%%%%%%%%%%%%%%%%%%%%%%%%%%%%%%%%%%%%%%%%%%%%%%%%%%%%%%%%%%%%%%%%%%%%%%
\subsection{Morphological Dependence}\label{sec:EW-Mrph}

The strength of the observed metal absorption signatures in the CGM can be
explored as a function of the azimuthal angle $\phi$ between the l.o.s. and
the projected major/minor axis of the
\textit{foreground} galaxy. Here we define the azimuthal angle ($\phi$) as the
projected angle between the \textit{background} galaxy l.o.s. and the
projected minor axis of the \textit{foreground} galaxy. Small azimuthal
angles($\phi\sim0^{\circ}$) refer to l.o.s. passing along the projected
minor axis of the \textit{foreground} galaxy, while large azimuthal angles
($\phi=90^{\circ}$) refer to l.o.s. passing along the projected major axis
of the \textit{foreground} galaxy. The effective radius ($r_{\rm
{eff, circ}}=r_{\rm{eff}}\sqrt{q}$ 
\footnote{$q$ is the axis ratio (b/a) of the elliptical isophotes that best
fit the galaxy.}) was obtained by fitting a single S\'ersic profile with no
constraints in the parameters and then was circularized using the galaxy
ellipticity \citep{Ribeiro16}. Alongside the effective radius, {\sc Galfit}
provides other structural parameters: S\'ersic index (n), the axis ratio of
the ellipse (b/a) and the position angle, $\theta_{\rm{PA}}$. Out of the 238
galaxies evaluated in this paper, only 97 are in the COSMOS field, covered
by HST imaging, and therefore have morphological parameters. To inspect the
dependency of the LIS-HIS absorption strengths with the azimuthal angle
($\phi$), we split our sample into three $\phi$ bins: [0-30]$^{\circ}$,
[30-60]$^{\circ}$ and [60-90]$^{\circ}$. Figure~\ref{Fig:EW-Pfl-Bg-PRP-ALL}
(lower panels) reports the equivalent width of different spectral features
in bins of $r_{\rm{eff}}$ and $\phi$.

To check and assess the robustness of possible correlations between the
morphological parameters and the equivalent widths measured in the composite
spectra for LIS, HIS, and Ly$_{\upalpha}$, we ran a Kendall-Tau rank test
between $W_{0}$ and $\phi$/$r_{\rm{eff}}$. For lines detected in only two
bins  the correlation can not be assessed, therefore we estimated the
significance of the difference between their absorption strength ($W_
{0}$). To do so, we applied a Student's t-test to determine the probability
that the $W_{0}$ variations between the low- and high $r_{\rm
{eff}}$/$\phi$ populations are not statistically significant. According to
the Student's statistics ($t$) a large $p$-value indicates that a large
probability of the null hypothesis to be correct (the two samples are not
statistically significant); a small $p$-value suggests that the difference is
significant. Both, Kendall-Tau and Student's t-test results are quoted in
Figure \ref{Fig:EW-Pfl-Bg-PRP-ALL} in the corresponding panels. 

Regarding the effective radius ($r_{\rm{eff}}$) of the galaxy, the Kendall-Tau
test shows a mild anti-correlation with C\,{\sc ii} ($p<0.117$), while it
does not show any significant correlation with Si\,{\sc ii} and Si\,{\sc iv}
($p>$0.602). We find a significant difference ($p<0.01$) in the $W_
{0}$ variations between small and large $r_{\rm{eff}}$ as shown by the
Student's t-test. However, no correlation is found for Ly$\upalpha$. This
suggests that C\,{\sc iv} gas is usually located in the CGM of larger
galaxies while C\,{\sc ii} gas is located in the CGM  of smaller galaxies.
Concerning the azimuthal angle, we do not find any significant correlation
with $\phi$ in any of the spectral lines inspected, in agreement with what
has been reported by \citep{Chen21} who inspected the azimuthal dependence of
Ly${\upalpha}$ emission of 59 star-forming galaxies($z_{med}\sim 2.3$). This
is opposed to the low-redshift scenario, where galaxies show high-velocity
biconical outflows oriented along the minor-axis and accreting material along
the major-axis. This result could be linked to the metal distribution along
the disk of star-forming galaxies as metal absorption systems are often
compact and poorly mixed, where cool low-ionization metal absorbers have
typical sizes of $\sim 1$ kpc \citep{Faucher-Giguere17}, while
high-ionization gas seen in absorption arises in multiple, extended
structures spread over $\sim$100 kpc \citep{Churchill15,Peeples19}
%%%%%%%%%%%%%%%%%%%%%%%%%%%%%%%%%%%%%%%%%%%%%%%%%%%%%%%%%%%%%%%%%%%%%%%%%%%%%%%
%%%%%%%%%%%%%%%%%%%%%%%%%%%%%%%%%RES-MORPHO%%%%%%%%%%%%%%%%%%%%%%%%%%%%%%%%%%%%
%%%%%%%%%%%%%%%%%%%%%%%%%%%%%%%%%%%%%%%%%%%%%%%%%%%%%%%%%%%%%%%%%%%%%%%%%%%%%%%

%%%%%%%%%%%%%%%%%%%%%%%%%%%%%%%%%%%%%%%%%%%%%%%%%%%%%%%%%%%%%%%%%%%%%%%%%%%%%%%
%%%%%%%%%%%%%%%%%%%%%%%%%%%%%%%%%%RES-CII/CIV%%%%%%%%%%%%%%%%%%%%%%%%%%%%%%%%%%
%%%%%%%%%%%%%%%%%%%%%%%%%%%%%%%%%%%%%%%%%%%%%%%%%%%%%%%%%%%%%%%%%%%%%%%%%%%%%%%
\subsection{C\,{\sc ii}/C\,{\sc iv} line ratio}\label{sec:LR}

A further inspection of the differences between LIS and HIS on the different
galaxy parameters come from the C\,{\sc ii}/C\,{\sc iv} equivalent width line
ratio. We focus on Carbon ions as they provide stronger absorptions with
higher S/N compared to aluminium or silicate ones. Figure~\ref
{Fig:EW-C2C4LR} shows C\,{\sc ii}/C\,{\sc iv} line ratio as a function of
impact parameter, stellar mass, and star formation rate. C\,{\sc ii}/C\,
{\sc iv} line ratio appears to anticorrelate with impact parameter
($b$), stellar mass and star formation rate. Similarly to the analyses for
individual line absorptions, we ran a Kendall-Tau to assess possible
correlations  for C\,{\sc ii}/C\,{\sc iv} line ratio and included it in all
panels of Figure~\ref{Fig:EW-C2C4LR}. We confirm that the anticorrelations
between C\,{\sc ii}/C\,{\sc iv} ratio and impact parameter, stellar mass, and
star formation rate are statistically robust. This suggests that C\,{\sc ii}
is more important than C\,{\sc iv} in the inner regions of these star-forming
galaxies, while the opposite occurs in the outskirts at large separations. 

On the other hand, star-forming galaxies and low-stellar mass with low-star
formation rates show a higher C{\sc ii}/C{\sc iv} line ratio compared with
galaxies that have high stellar mass and high star formation rates. Our
results suggest that galaxies with higher star formation rates and large
stellar masses are capable to sweep out the highly ionized gas (traced by C\,
{\sc iv}) farther away from the galaxy compared with less star-forming and
less massive galaxies. Another possible explanation is that more active and
massive galaxies have stronger ionizing fluxes, able to ionize gas at larger
distances compared with less massive and less star-forming galaxies. Recent
studies have shown that fast and energetic outflows can push material away
from the central regions in star-forming galaxies, more effectively in
galaxies with large SFR (with a weaker dependence on stellar mass; 
\citealt{Heckman15,Cicone16,Trainor15,Feltre20}.) The presence of an AGN can
make this effect even more dramatic, as shown by an enhanced gas budget in
{\sc H\, i} and low-ionization gas in high-redshift ($z>2$) galaxies \citep
{Prochaska14}, suggesting larger accretion rates or a net gain of cold gas
in the CGM in these star-forming galaxies \citep{Faucher16}. However, the
physical mechanism through which the AGN removes and/or heats the gas and
suppresses accretion is not clear \citep{Tumlinson17}. Determining the
nature of the source responsible for ionizing the gas in the CGM is crucial
to improve our understanding of the multi-phase CGM, but is beyond the scope
of this paper. 

%%%%%%%%%%%%%%%%%%%%%%%%%%%%%%%%%%%%FIG-LR%%%%%%%%%%%%%%%%%%%%%%%%%%%%%%%%%%%%%
%%%%%%%%%%%%%%%%%%%%%%%%%%%%%%%%%%EW-vs-REF%%%%%%%%%%%%%%%%%%%%%%%%%%%%%%%%%%%%
   \begin{figure*}
   \centering
   \includegraphics[width=18cm]{LRC2ToC4LR-PRP-New4GEX.pdf}
   \caption{C{\sc ii}/C{\sc iv} $W_{0}$ line ratio as a function of the impact
    parameter $b$, and the \textit{foreground} galaxy's stellar mass and SFR.
    C{\sc ii}/C{\sc iv} $W_{0}$ were obtained considering the exact same set
    of \textit{fg-bg} galaxy pairs. Average and median $W_{0}$ are shown in
    solid-red and open-blue symbols, solid black symbols
    correspond to upper limits with SNR <3. Error bars correspond to
    1$\sigma$ confidence intervals for average or median values based on a
    Bootstrap analysis.}
    \label{Fig:EW-C2C4LR}%
    \end{figure*}
%%%%%%%%%%%%%%%%%%%%%%%%%%%%%%%%%%EW-vs-REF%%%%%%%%%%%%%%%%%%%%%%%%%%%%%%%%%%%%
%%%%%%%%%%%%%%%%%%%%%%%%%%%%%%%%%%%%FIG-LR%%%%%%%%%%%%%%%%%%%%%%%%%%%%%%%%%%%%%
%%%%%%%%%%%%%%%%%%%%%%%%%%%%%%%%%%%%%%%%%%%%%%%%%%%%%%%%%%%%%%%%%%%%%%%%%%%%%%%
%%%%%%%%%%%%%%%%%%%%%%%%%%%%%%%%%%RES-CII/CIV%%%%%%%%%%%%%%%%%%%%%%%%%%%%%%%%%%
%%%%%%%%%%%%%%%%%%%%%%%%%%%%%%%%%%%%%%%%%%%%%%%%%%%%%%%%%%%%%%%%%%%%%%%%%%%%%%%

%%%%%%%%%%%%%%%%%%%%%%%%%%%%%%%%%%%%%%%%%%%%%%%%%%%%%%%%%%%%%%%%%%%%%%%%%%%%%%%
%%%%%%%%%%%%%%%%%%%%%%%%%%%%%%%%RES-Ly-alpha-Emm%%%%%%%%%%%%%%%%%%%%%%%%%%%%%%%
%%%%%%%%%%%%%%%%%%%%%%%%%%%%%%%%%%%%%%%%%%%%%%%%%%%%%%%%%%%%%%%%%%%%%%%%%%%%%%%
\subsection{Ly$\upalpha$-emission}\label{sec:Ly-Emm}

As we already noted in section~\ref{sec:EW-Rad} (Figure~\ref
{Fig:Lns-Bg-Sep-4}), the Ly$\upalpha$ emission component is clearly detected
in our composite spectra. Figure \ref{Fig:EW-Lya-Emm} shows the Ly$\upalpha_
{\rm{em}}$ rest equivalent width as a function of the impact parameter. We
find that Ly$\upalpha_{\rm{em}}$ decreases as a function of the impact
parameter  $b$ similarly to what \cite{Chen21} reported from 2D Ly$\upalpha_
{\rm{em}}$ maps. We also explore the Ly$\upalpha_{\rm{em}}$ equivalent width
strength as a function of the galaxy stellar mass, star formation rate and
azimuthal angle. However, we do not find any significant correlation in these
cases. This might be caused by the fact that at $z\sim$2 the Ly$\upalpha_{\rm
{em}}$-stellar mass relation is weaker compared with galaxies at higher
redshifts \citep{Du18}. Another possibility is the fact that our Ly$\upalpha_
{\rm{em}}$ measurements come from composite spectra, hence coming from
star-forming galaxies with and without direct Ly$\upalpha_{\rm
{em}}$ detections (EW$\leq0\AA$, EW$>0\AA$ and EW$\geq20\AA$ \citealt
{LeFevre15,Hathi16}), covering a wide range of stellar masses and star
formation rates that may have diluted a real Ly$\upalpha_{\rm{em}}$ signal
\citep{Steidel11}.

%%%%%%%%%%%%%%%%%%%%%%%%%%%%%%%FIG-Ly-alpha-Emm%%%%%%%%%%%%%%%%%%%%%%%%%%%%%%%%
%%%%%%%%%%%%%%%%%%%%%%%%%%%%%%%%%%EW-vs-REF%%%%%%%%%%%%%%%%%%%%%%%%%%%%%%%%%%%%
\begin{figure}
\centering
\includegraphics[width=8.5cm]{Lyalpha-Emm.pdf}
\caption{Equivalent width ($W_{0}$) for the Ly${\upalpha}$ emission as a
 function of the impact parameter ($b$) obtained from the average
 (solid red-line) and median (dot-dashed blue-line) composite spectra. Error
 bars correspond to 1$\sigma$ CIs for average or median values based on a
 Bootstrap analysis.}
\label{Fig:EW-Lya-Emm}%
\end{figure}  
%%%%%%%%%%%%%%%%%%%%%%%%%%%%%%%%%%EW-vs-REF%%%%%%%%%%%%%%%%%%%%%%%%%%%%%%%%%%%%
%%%%%%%%%%%%%%%%%%%%%%%%%%%%%%%FIG-Ly-alpha-Emm%%%%%%%%%%%%%%%%%%%%%%%%%%%%%%%%

%%%%%%%%%%%%%%%%%%%%%%%%%%%%%%%%%%%%%%%%%%%%%%%%%%%%%%%%%%%%%%%%%%%%%%%%%%%%%%%
%%%%%%%%%%%%%%%%%%%%%%%%%%%%%%%%RES-Ly-alpha-Emm%%%%%%%%%%%%%%%%%%%%%%%%%%%%%%%
%%%%%%%%%%%%%%%%%%%%%%%%%%%%%%%%%%%%%%%%%%%%%%%%%%%%%%%%%%%%%%%%%%%%%%%%%%%%%%%

%%%%%%%%%%%%%%%%%%%%%%%%%%%%%%%%%%%%%%%%%%%%%%%%%%%%%%%%%%%%%%%%%%%%%%%%%%%%%%%
%%%%%%%%%%%%%%%%%%%%%%%%%%%%%%%%%%%%RESULTS%%%%%%%%%%%%%%%%%%%%%%%%%%%%%%%%%%%%
%%%%%%%%%%%%%%%%%%%%%%%%%%%%%%%%%%%%%%%%%%%%%%%%%%%%%%%%%%%%%%%%%%%%%%%%%%%%%%%

%%%%%%%%%%%%%%%%%%%%%%%%%%%%%%%%%%%%%%%%%%%%%%%%%%%%%%%%%%%%%%%%%%%%%%%%%%%%%%%
%%%%%%%%%%%%%%%%%%%%%%%%%%%%%%%%%DISCUSSION%%%%%%%%%%%%%%%%%%%%%%%%%%%%%%%%%%%%
%%%%%%%%%%%%%%%%%%%%%%%%%%%%%%%%%%%%%%%%%%%%%%%%%%%%%%%%%%%%%%%%%%%%%%%%%%%%%%%
\section{Discussion}\label{sec:Dis}

Analyses of low and high-ionization state absorption features and their
dependence on stellar mass, star formation rate, and galaxy inclination have
been widely explored at $\lesssim0.5$ (see \citealt
{Tumlinson17,Kacprzak17}). At high-redshift however, the studies of the CGM
are limited. %compared to their low-redshift counterparts.  
\cite{Steidel10} used 512 close ($<15\arcsec=124$kpc) angular pairs of
$z\sim$2--3 ($z_{bg}\sim$2.3) galaxies to map the cool and diffuse gas
around galaxies. They found strong evidence suggesting the presence of
superwind outflows and proposed a simple model of the gas in the CGM, where
cool gas is distributed symmetrically around every galaxy, accelerating
radially outwards with the outflow velocity increasing with radius. Later,
\cite{Turner14} and \cite{Lau16} reported detections that suggest that in
galaxies at $\langle z \rangle$\,=\,2.4 the CGM extends at least up to
180\,kpc. Moreover, the mass of metals found within the halo is substantial
and equivalent to $\sim$25\% of the metal mass within the interstellar
medium \citep{Rudie19}. This is, however, a lower percentage than what has
been reported from studies in low-redshift galaxies \citep{Werk14}, that
suggests a considerable redistribution of the metal content of galaxies in
an inside-out fashion over the last $\sim$8.5 Gyr. We note that eight of the
nine resonance Si\,{\sc ii} transitions fall within the
[900-1900]$\AA$ wavelength range considered here: $\uplambda 989$,
$\uplambda 1020$ , $\uplambda\uplambda 1190-1193$, $\uplambda 1260$,
$\uplambda 1304$ $\uplambda 1526$, and $\uplambda 1808 \AA$ \citep
{Shull81}. Lines located blueward to Ly$_{\upalpha}$ are not included in our
analyses, as the result of the low coverage of our composite spectra at this
wavelength range and because UV continuum level blueward to Ly$_
{\upalpha}$ is particularly sensitive to the dust content, metallicity, and
the age of the stellar population \citep{Trainor15} and their strength could
be affected by IGM absorptions \citep{Shapley03}. The $\uplambda1304$
component is blended with  O\,{\sc i} $\uplambda1302$ absorption line and is
close to excited fine-structure emission transitions that could lower $W_
{0}$ measurements \citep{Trainor15}. Finally,  $\uplambda1526$ is the least
resonance component affected by blends \citep{Shapley03} and is the
strongest Si\,{\sc ii} line amongst their counterparts. In this work, we
chose $\uplambda 1526 \AA$ for our Si\,{\sc ii} $W_{0}$ computations as this
component is free of blending effects caused by  the low spectral resolution
of our dataset. However, as noted by \cite{Jones18} Si\,{\sc ii} transitions
probe diverse optical depths, and hence the most accurate way to compute
Si\,{\sc ii} absorption strength is to consider the contributions of all the
Si\,{\sc ii} resonance components ($\uplambda\uplambda\uplambda 1260, 1304,
1526\AA$) within the analysed range, by averaging their strengths and
deblending them from nearby spectral features that could affect their
individual $W_{0}$ measurements. Nevertheless, different studies have
adopted diverse strategies to compute the LIS (including Si\,{\sc ii}
resonance components) $W_{0}$ strength. For example, \cite{Jones18} averaged
Si\,{\sc ii} $\uplambda\uplambda\uplambda 1260, 1304, 1526\AA$, O\,{\sc i}
$\uplambda 1302$, C\,{\sc ii} $\uplambda 1334$, Al\,{\sc ii} $\uplambda
1670$, Fe\,{\sc ii} $\uplambda 2382$ and Mg\,{\sc ii} $\uplambda\uplambda
2796-2803$ components, 
\cite{Ranjan22} considered the average of the eight Si\,{\sc ii} resonance
components to estimate the $W_{0}$ Si\,{\sc ii}  absorption in high-z Damped
Lyman $\upalpha$ (DLA) systems, \cite{Du16} averaged $\uplambda 1526$, Si\,
{\sc ii}, Al\,{\sc ii} $\uplambda 1670$ Ni\,{\sc ii}
$\uplambda\uplambda1741-1751$, Si\,{\sc ii} $\uplambda 1808$, and Fe\,
{\sc ii} $\uplambda 2382 \AA$ absorption lines to compute an overall LIS
average absorption estimation, while later-on \cite{Du18} considered
$\uplambda 1260$, O\,{\sc i}-Si\,{\sc ii} $\uplambda\uplambda1302-1304$, C\,
{\sc ii} $\uplambda 1334$, and Si\,{\sc ii} $\uplambda 1526 \AA$, for
similar computations. The differences in the number of Si\,{\sc ii}
resonance components considered for $W_{0}$ computations could be the reason
why similar Si\,{\sc ii} trends as those of C\,{\sc ii} with SFR and stellar
mass  shown in Fig. \ref{Fig:EW-Pfl-Bg-PRP-ALL} are not appreciated. As
shown by \cite{Jones18} different low-ions transitions(Si\,{\sc ii} , Fe\,
{\sc ii}, and Ni\,{\sc ii}) show non-uniform covering fractions in the CGM
of star-forming galaxies at high redshift ($z\simeq2-3$).

In this work we have used a sample of 238 galaxy close pairs to probe the CGM
around star-forming galaxies at $z\sim2.6$. Our results show an
anticorrelation between the equivalent width of the absorption lines and the
impact parameter ($b$) for this high redshift star-forming galaxy sample and
are consistent with previous results at lower redshift. Here we detect
Ly$\upalpha$/LIS and HIS absorption at distances up to 172\,kpc and 146\,kpc. 
Our low-ionization state (C\,{\sc ii}, Si\,{\sc ii}), and
high-ionization state (C\,{\sc iv} Si\,{\sc iv}) absorption lines detections
shape an upper envelope of the equivalent width distribution coming from
studies at low redshift (see Figure \ref{Fig:EW-Pfl-Bg-sep-4}). This is
further illustrated by the differences in the slopes of the $W_{0}$ radial
profiles between low- and high-ionization state absorptions of the same ion
species, where Si\,{\sc ii}--Si\,{\sc iv} and C\,{\sc ii}--C\,{\sc iv}
absorption line pairs are different at $\geq5\sigma$ level. Moreover, when
compared with galaxies at $z\sim2.3$ \citep{Steidel10} our results show a
significant difference in their slopes at a 5$\sigma$ level for Si\,{\sc ii}
and 3$\sigma$ level for C\,{\sc ii} and Al\,{\sc ii} absorption lines. These
discrepancies between LIS and HIS suggest that, within the CGM, cold (T$<10^
{4.5}$K) and dense gas is more extended in galaxies at $z\sim2.6$ compared
with galaxies at $z\sim2.3$ and lower redshifts, as probed by Si\,{\sc ii},
C\,{\sc ii}, Si\,{\sc iv} C\,{\sc iv}. These results suggests a potential
redshift evolution for the CGM gas content that produce these absorptions. As
higher covering factors of neutral gas lead to stronger absorption
lines \citep{Du18}, this suggests a higher content of neutral gas in high
redshift galaxies compared with low-redshift galaxies. This concentration of
neutral gas could represent a reservoir of gas that can be later accreted
onto the galaxies to fuel future star formation \citep{Hafen19,Hafen20}.

We note that it is possible to increase the maximum angular projected
separation($\geq23\arcsec$) allowing us to probe larger distances closer to
the CGM and IGM boundary following \cite{Chen20}. However, we have limited
our analyses to 23$\arcsec\sim$170kpc and galaxy spectra with highly reliable
redshifts to guarantee detection with good S/N. Spectra with unreliable
redshift measurements could wash out the expected metal absorptions and/or
produce spurious features in the final composite spectra. This is further
supported by the observed metallicity dependence with impact parameter, where
LIS are preferably located in the inner CGM (close to the
galaxy; \citealt{Lau16}) while HIS (e.g. O\,{\sc vi} are expected to dominate
the CGM at larger separations \citealt{Shen13}) making it difficult to detect
weak LIS absorptions at larger separations given their expected $W_{0}$.

%%%%%%%%%%%%%%%%%%%%%%%%%%%%%%%%%%%%%%%%%%%%%%%%%%%%%%%%%%%%%%%%%%%%%%%%%%%%%%%
%%%%%%%%%%%%%%%%%%%%%%%%%%%%%%%%%DISC-MASS-SFR%%%%%%%%%%%%%%%%%%%%%%%%%%%%%%%%%
%%%%%%%%%%%%%%%%%%%%%%%%%%%%%%%%%%%%%%%%%%%%%%%%%%%%%%%%%%%%%%%%%%%%%%%%%%%%%%%
\subsection{LIS/HIS absorption dependencies.}

At $\lesssim1$, \cite{Churchill00b} explored a sample of 45 Mg\,{\sc ii}
absorption systems in high-resolution QSO spectra, and studied their spectra,
together with the C\,{\sc iv} and Fe\,{\sc ii} absorption profiles,
suggesting that the gas responsible for Mg\,{\sc ii} and C\,{\sc iv}
absorptions arises from gas in different phases, (i.e. gas producing HIS
absorption features have typical temperatures T>10$^{5}$K \citealt
{Tumlinson17}). Moreover, they observed an evolution of the absorbing gas,
that is consistent with scenarios of galaxy evolution in which mergers and
accretion of "protogalactic clumps" provide gas reservoirs responsible for
the elevated star formation activity at high-redshift, while at intermediate
and lower redshifts ($z\leq$1), the balance of high- and low-ionization state
CGM gas may be related to the presence of star-forming regions in the host
galaxy \citep{Churchill00b}. Later on, \cite{Zibetti07} found that
early-type (quiescent, red) galaxies ($0.37 < z < 1$) are associated with
weaker  Mg\,{\sc ii} absorptions while stronger systems are located in
late-type (star-forming, blue) galaxies. These results provided evidence that
the SFR correlates with Mg\,{\sc ii} absorption strength caused by the
presence of outflows from star-forming/bursting galaxies.  This scenario is
also supported by subsequent studies that demonstrated that SFR correlates
with outflow velocity deduced from the Mg\,{\sc ii} absorption strength 
\citep{Weiner09, Martin12,Tremonti07,Rubin10}, and by the correlation between
the stellar mass and Ly$\upalpha$ absorption strength \citep
{Bordoloi18,Wide21}.

Our detection of LIS and HIS absorption lines suggests that at $\langle
z \rangle \sim 2.6$ C\,{\sc ii} and C\,{\sc iv} features are correlated with
star formation rate and stellar mass. C\,{\sc ii} is stronger in galaxies
with low star formation rates and low stellar masses, while C\,{\sc iv} is
stronger in galaxies with high SFR and large stellar masses. These results
are consistent with what has been reported for low-redshift galaxies in
scenarios where LIS and HIS absorbers have different spatial distributions.
Low-ionization state metal absorbers probe the gas that is less ionized than
high-ionization state metal absorbers \citep{Nagao06} and thus they are
expected to be located in gas regions with different density
conditions \citep{Burchett16}. Moreover, studies of low-redshift galaxies
($z\leq0.5$) have shown that as a consequence of the interaction between a
starburst-driven wind and the preexisting CGM at radii as large as 200\,kpc,
the CGM around star-forming galaxies with high star formation rates differs
systematically compared to galaxies with lower SFRs, as probed  by
Ly$\upalpha$, Mg\,{\sc ii}, Si\,{\sc ii}, C\,{\sc iv}, and O\,{\sc vi}
absorption lines \citep{Tumlinson11a,Borthakur13,Lan14,Heckman17}. However,
as suggested by 
\cite{Cicone16} and \cite{Gatkine22}, their correlation with stellar mass and
star-formation rate in fact might be a consequence of a main-sequence
offset, rather than simply a correlation with star-formation rate or stellar
mass  and as LIS/HIS covering factors can be affecetd by environmental
processes  \citep{Dutta21}. Hence, caution should be taken when interpreting
LIS/HIS correlations with SFR and/or stellar mass.

The origin of the reservoir of cold neutral gas around low stellar mass
galaxies and low SFRs at these distances($\langle b \rangle\approx 120$kpc
see Appendix \ref{sec:AppendixTbl}) is uncertain. One explanation come
from a soft radiation field unable to ionize HIS (i.e. C\,{\sc iv}, Si\,
{\sc iv}), as the production of C\,{\sc iv}, Si\,{\sc iv} absorption features
in the CGM, requires photons with energies >45 eV associated with hard
ionizing radiation fields from massive stars, AGN, and radiative shocks \cite
{Feltre20,Trainor15}. In fact as shown by \cite{Gatkine22} HIS are more
affected by high-velocity outflows, they show a strong correlation with SFR
and are ubiquitous in high-SFR systems. Another possible explanation comes
from the galactic fountain scenario, predicted by cosmological galaxy
formation simulations \citep{Oppenheimer10,Vogelsberger13}, and where the
presence of cold low-ionization gas result from metal-rich gas ejected in
previous star-formation episodes that fall back to the disk \citep
{Fraternali06,Hobbs15}. Moreover, we know that stellar winds are dominant in
the inner region of the CGM <60kpc \citep{Chen20}, and depending on their
velocity, stellar winds in low mass galaxies might be able to expel material
or be reaccreted as recycled material \citep{SanchezA17}. However, if the
velocity outflow is insufficient to eject material out into the IGM, the
recycle time scales in low-mass galaxies in which gas returns to the galaxy
can be small as those in high-mass systems 
\citep{vandeVoort17}. In fact, \cite{Martin12} detected Fe\,{\sc ii} Doppler
shifts and the $V_{max}$-Mg\,{\sc ii} values that suggest that outflows
reach the circumgalactic medium with Mg\,{\sc ii} absorption at blueshifts
as high as 700 km s$^{-1}$, reaching 70 kpc in 100 Myr, a short enough time
for the host galaxy to sustain SFR even if the SFR declines due to an
outflow. Nevertheless, our measurements do not allow us to break-down the
multiple components of the absorption features in our foreground galaxy
("down-the-barrel") composite spectra and detect blue-shifted offsets that
could be related to high-velocity outflows. We note that it is possible to
combine down-the-barrel and QSO/galaxy-galaxy pairs to unambiguously detect
blue-shifted absorptions relative to the galaxy systemic velocity, and
quantify independently the main properties of the detected outflow 
\citep{Kacprzak14,Bouche16,Lehner17}.

Fig \ref{Fig:EW-Pfl-Bg-PRP-ALL} (lower panels) shows the dependence of line
absorption strength with galaxy's effective radius ($r_{\rm{eff}}$) and
azimuthal angle ($\phi$). Concerning the dependence of the absorption
strength ($W_{0}$) with the size of the galaxy given by the effective radius
($r_{\rm{eff}}$), we find that LIS are stronger in smaller galaxies compared
to HIS absorptions that are shown to be stronger in larger galaxies, in
agreement with \cite{Rudie19} who reported that LIS absorption occurs
spatially closer to the galaxy, and gas producing HIS absorptions can be
located well beyond the virial radius. Regarding the dependence with
azimuthal angle, we do not find any significant correlation with $\phi$ for
both LIS and HIS. However, these results are based on the fraction of
galaxies for which we have morphological measurements ($\sim 40\%$ of the
parent sample). Thus, these results have to be confirmed by a similar
morphological analysis using larger samples at these high redshifts, as
galaxy orientation plays a major role in the measured strength of LIS and
HIS. This has been shown at low-redshift from studies using down the
barrel \citep{Rubin14,Bordoloi14b} and QSO-galaxy pairs \citep
{Kacprzak11b,Bordoloi11} analyses. Moreover, at z $\gtrsim2$ star-forming
galaxies have clumpy morphologies \citep{Foster11}, with galactic winds that
are mainly driven by outflows from prominent star-forming clumps \citep
{Genzel11}, and have not formed yet a stable disk (if any) capable of
collimate galactic winds into bipolar outflows \citep{FaucherG17}, indicating
that the minor/major axis dichotomy, associated with rotation present at
low-redshift (z<1) is not broadly applicable at z$\sim$2-3 
\citep{Law12,Nelson19,Price20}. Therefore, it is not obvious that at
high-redshift ($z\gtrsim2$) the azimuthal angle can be determined and, if it
can, it has any physical meaning.

The dependence of LIS and HIS absorption strength on the azimuthal angle has
been explored in the literature to prove/disprove theories on how galaxies
accrete gas from the intergalactic medium. Inflowing material along filaments
in the IGM is expected to be the largest source of accretion \citep
{Keres05}. Simulations have shown that accretion of metal-poor gas inflows
occurs along the major axis of galaxies, where outflows are preferably
located along the semi-minor axis form bipolar outflows \citep{Putman17}, and
gas being accreted onto galaxies that later can trigger star formation has
been supported from evidence collected by multiple CGM studies at
low-redshift \citep
{Kacprzak10,Kacprzak12a,Kacprzak15a,Kacprzak19,Lan18,Martin19a,Nielsen15,
Rubin18b,Zhu13a,Tumlinson11a,Bordoloi14a,Lan14}. Recent cosmological
hydrodynamical simulations examine the physical properties of the gas located
in the CGM of star-forming galaxies as a function of angular
orientation \citep{Peroux20b}. They found that the CGM varies strongly with
impact parameter, stellar mass, and redshift. Moreover, they suggest that the
inflow rate of gas is more substantial along the galaxy major axis, while the
outflow is strongest along the minor axis. 

The correlations between  C\,{\sc ii} and C\,{\sc iv} with impact parameter,
star formation rate, and stellar mass scenario of a multi-phase CGM where LIS
absorptions are produced by denser gas located closer to the galaxy, whether
this dense neutral gas component is part of an outflow or material falling
back to the galaxy taking part of a galactic fountain that eventually could
be funnelled to the galaxy to sustain star formation activity \citep
{Keres05} is not clear and can not be assessed by our datasets. On the other
hand, HIS absorption features associated with strong stellar winds produced
by high star formation activity capable to ionize radiation in high stellar
mass galaxies with high star formation activity sweeping out material to the
outer regions of these galaxies \citep
{Putman17,Bower16,Oppenheimer16,Voit15b}. 

Similar analyses including observations at different redshifts coming from
different complementary surveys  (e.g. VANDELS \citep
{McLure18,Pentericci18}, zCOSMOS \citep{Lilly07,Lilly09}, VVDS\citep
{LeFevre13b}, DEIMOS10K \citep{Hasinger18}), can help to increase the number
of galaxy pairs with close angular separations($b<6\arcsec$), deblend close
spectral line features (e.g. C \,{\sc ii}-O \,{\sc iv} $\uplambda\uplambda$
1334.5-1343.0 $\AA$; Si \,{\sc iv} $\uplambda\uplambda$ 1393.8-1402.8 $\AA$),
and cover different low- and high-ionization state lines(e.g.  O \,{\sc vi}
$\uplambda\uplambda$ 1032-1038 $\AA$,  Mg \,{\sc ii} $\uplambda$ 2798 $\AA$).
Additionally, as it has been shown AGN activity is dependent on stellar mass
and SFR (e.g. \citealt{Lemaux14,Bongiorno16,Magliocchetti20}), thus, studies
considering \textit{bg}/QSO$-$AGN galaxy pairs \citep
{Hennawi06,Prochaska14} are needed to study the effect that the presence of
an AGN has on feedback and quenching in star-forming galaxies at low- and
high-redshifts.
%%%%%%%%%%%%%%%%%%%%%%%%%%%%%%%%%%%%%%%%%%%%%%%%%%%%%%%%%%%%%%%%%%%%%%%%%%%%%%%
%%%%%%%%%%%%%%%%%%%%%%%%%%%%%%%%%DISC-MASS-SFR%%%%%%%%%%%%%%%%%%%%%%%%%%%%%%%%%
%%%%%%%%%%%%%%%%%%%%%%%%%%%%%%%%%%%%%%%%%%%%%%%%%%%%%%%%%%%%%%%%%%%%%%%%%%%%%%%

%%%%%%%%%%%%%%%%%%%%%%%%%%%%%%%%%%%%%%%%%%%%%%%%%%%%%%%%%%%%%%%%%%%%%%%%%%%%%%%
%%%%%%%%%%%%%%%%%%%%%%%%%%%%%%%%%%DISC-CII/CIV%%%%%%%%%%%%%%%%%%%%%%%%%%%%%%%%%
%%%%%%%%%%%%%%%%%%%%%%%%%%%%%%%%%%%%%%%%%%%%%%%%%%%%%%%%%%%%%%%%%%%%%%%%%%%%%%%
\subsection{C\,{\sc ii} / C\,{\sc iv} line ratio.} 

A further inspection of the differences between LIS and HIS on the different
galaxy parameters explored comes from C\,{\sc ii}/C\,{\sc iv} line ratio. We
found that C\,{\sc ii}/C\,{\sc iv} line ratio anti-correlates with impact
parameter, stellar mass, and star formation rate.  This implies that C\,
{\sc iv} is mostly located at larger distances in more massive
galaxies with higher star formation rates, while C\,{\sc ii} dominates at
smaller impact parameters and is found mostly in low stellar mass
galaxies with low star formation rates. These results further support the
picture of a multi-phase CGM where LIS line absorptions are produced by
denser gas with lower temperatures (T$<10^{4.5}$K) that is located close to
the central galaxy while HIS line absorptions are produced by warm gas
located at larger distances from the central galaxy. Simulations have shown
that much of the ongoing gas accretion occurs towards the edges of the
galaxies to avoid the dominant feedback(outflows) from the central regions
\citep{Stewart12,Fernandez12,Putman17}, however, we do not find supporting
evidence of the low-redshift scenario where low-ionization state gas infalls
along the major axis of star-forming galaxies, accompanied by large-scale
outflows along the minor axis forming bipolar outflows. On the contrary, our
results suggest that this scenario is not broadly applicable at z$\sim$2. We
note that, we have also explored Si\,{\sc ii} / Si\,{\sc iv} and  Al\,
{\sc ii} / Al\,{\sc iii} equivalent width line ratios, however, we did not
detect any significant correlations. On one hand, Si\,{\sc iv} predominantly
arises in denser gas closer to galaxies more similar to other LIS state
lines such as Mg\,{\sc ii} than to that of higher ions \citep
{Ford13,Ford14}, and hence Si\,{\sc ii} and Si\,{\sc iv} line absorption
might be probing gas with similar physical conditions. On the other, Al\,
{\sc iii} is an intermediate-ionization state(IIS) tracer of moderately
photoionized warm gas generally associated with neutral phase gas \citep
{Savage01,Knauth03,Vladilo01}. In fact, Al\,{\sc iii} absorption is
associated with low-ions as shown by the significant correlation between the
velocity widths of Al\,{\sc iii} and low-ionization species (e.g. Si\,
{\sc ii}, Fe\,{\sc ii}, Zn\,{\sc ii}, \citealt{Howk99b}), suggesting that a
substantial fraction of the low-ionization ions may be associated with
moderately ionized gas traced by Al\,{\sc iii}, Fe\,{\sc iii}
\citep{Wolfe05} and that, doubly ionized species (e.g. Al\,{\sc iii}, Fe\,
{\sc iii}, C\,{\sc iii}, Si\,{\sc iii}) often have comparable column
densities to the singly ionized (e.g. Al\,{\sc ii}, Fe\,{\sc ii}, C\,
{\sc ii}, Si\,{\sc ii}) ones \citep{Mas-Ribas17b}.

%%%%%%%%%%%%%%%%%%%%%%%%%%%%%%%%%%%%%%%%%%%%%%%%%%%%%%%%%%%%%%%%%%%%%%%%%%%%%%%
%%%%%%%%%%%%%%%%%%%%%%%%%%%%%%%%%%DISC-CII/CIV%%%%%%%%%%%%%%%%%%%%%%%%%%%%%%%%%
%%%%%%%%%%%%%%%%%%%%%%%%%%%%%%%%%%%%%%%%%%%%%%%%%%%%%%%%%%%%%%%%%%%%%%%%%%%%%%%
\subsection{Ly$\upalpha$-emission.}

Regarding Ly$\upalpha_{\rm{em}}$, several studies have been carried out to
explore the relation with galaxy properties. At low-redshift \cite
{Runnholm20}, used the Lyman Alpha Reference Sample (LARS) to obtain
correlations between Ly$\alpha$ and different galaxy properties (i.e. star
formation rate, dust extinction, compactness, and the gas covering fraction).
At high-redshiftit has been found that bluer galaxies show stronger
Ly$\upalpha_{\rm{em}}$
\citep{Shapley03,Pentericci10,Berry12,Erb16}, and that low-stellar mass
galaxies with lower SFRs show stronger Ly$\upalpha_{\rm{em}}$ than
high-stellar mass galaxies with higher SFRs \citep
{Vanzella09,Stark10,Erb06a,Jones12,Shapley03,Kornei10,Hathi16}. Additionally
Ly$\upalpha_{\rm{em}}$ is correlated with LIS absorption: stronger LIS
absorptions correspond to weaker Ly$\upalpha_{\rm{em}}$, as high
concentrations of neutral gas responsible for strong LIS absorption are also
responsible for scattering out of the l.o.s. Ly$\upalpha$ photons \citep
{Shapley03,Vanzella09,Jones12}. More recently \cite{Du18},
\citep{Trainor19} and \cite{Pahl20} studied the spectroscopic properties of
star-forming galaxies at $z\sim$2-5 through composite spectra grouped
through different galaxy properties. They detected stronger Ly$\upalpha_{\rm
{em}}$ at higher redshift at fixed stellar mass, SFR, and UV Luminosity, and
found that the LIS-Ly$\upalpha_{\rm{em}}$ relation is redshift independent
suggesting that this is caused by the variations of the neutral gas covering
fraction favoring Ly$\alphaup$ escape and production and/or dust content in
the ISM and CGM. Moreover \cite{Oyarzun16,Oyarzun17} shown that at
high-redhfit ($3<z<4.6$) Ly$\upalpha_{\rm{em}}$ anticorrelations with
stellar mass, star formation rate, and UV luminosity are stronger in
low-stellar-mass populations explained by the rapidly increasing neutral gas
fraction of the universe at higher redshifts \citep{Du21}.

%%%%%%%%%%%%%%%%%%%%%%%%%%%%%%%%%%%%%%%%%%%%%%%%%%%%%%%%%%%%%%%%%%%%%%%%%%%%%%%
%%%%%%%%%%%%%%%%%%%%%%%%%%%%%%%%%%DISC-CII/CIV%%%%%%%%%%%%%%%%%%%%%%%%%%%%%%%%%
%%%%%%%%%%%%%%%%%%%%%%%%%%%%%%%%%%%%%%%%%%%%%%%%%%%%%%%%%%%%%%%%%%%%%%%%%%%%%%%


%%%%%%%%%%%%%%%%%%%%%%%%%%%%%%%%%%%%%%%%%%%%%%%%%%%%%%%%%%%%%%%%%%%%%%%%%%%%%%%
%%%%%%%%%%%%%%%%%%%%%%%%%%%%%%%%%DISCUSSION%%%%%%%%%%%%%%%%%%%%%%%%%%%%%%%%%%%%
%%%%%%%%%%%%%%%%%%%%%%%%%%%%%%%%%%%%%%%%%%%%%%%%%%%%%%%%%%%%%%%%%%%%%%%%%%%%%%%

%%%%%%%%%%%%%%%%%%%%%%%%%%%%%%%%%%%%%%%%%%%%%%%%%%%%%%%%%%%%%%%%%%%%%%%%%%%%%%%
%%%%%%%%%%%%%%%%%%%%%%%%%%%%%%%%%%CONCLUSIONS%%%%%%%%%%%%%%%%%%%%%%%%%%%%%%%%%%
%%%%%%%%%%%%%%%%%%%%%%%%%%%%%%%%%%%%%%%%%%%%%%%%%%%%%%%%%%%%%%%%%%%%%%%%%%%%%%%
\section{Conclusions}\label{sec:Con}

This paper presents stacks of 238 background galaxy spectra to probe the
CGM content and extent around star-forming galaxies at $\langle
z \rangle \sim 2.6$. We use only spectra with highly reliable spectroscopic
redshifts ($95-100\%$ probability  of being correct) to identify the
low-ionization state (LIS: Si\,{\sc ii}, C\,{\sc ii}) and high-ionization
state (HIS: C\,{\sc iv}, Si\, {\sc iv}) metal absorption lines, and constrain
their spatial distribution, and their dependence with stellar mass, star
formation rate, azimuthal angle, and effective radius.

We summarize our main results below:
   \begin{enumerate}
      %1
      \item We detect LIS and HIS metal absorption lines in the CGM around
       star-forming galaxies at distances up to 172\,kpc  and 146\,kpc.
       The limitations owing to the size of our sample do not
       allow us to follow metal lines beyond these distances. The strength of
       these absorptions decreases at increasing distances from the galaxy,
       consistently with results published in the literature. At any fixed
       distance from the galaxy, the strength of all absorption lines that we
       identify in our sample at $z \sim 2.6$ is larger than any other
       measurement at lower redshift, providing evidence of a redshift
       evolution of the CGM gas content responsible for producing these
       absorptions.
       %2
      \item We do not find any significant correlation between the LIS/HIS
       absorptions with the azimuthal angle ($\phi$). This is opposed to the
       scenario at low-redshift where, cold gas (traced by LIS line) is
       infalling onto galaxies along the plane containing the disk, while the
       gas heated and processed by star-formation (traced by HIS lines) is
       outflowing perpendicularly to the plane. This can be explained by the
       fact that high-redshift galaxies have not formed a stable disk capable
       of collimating galactic winds into bipolar outflows. However, due to
       the small sample of close pairs with available morphological features,
       these trends need to be confirmed applying a similar analysis on
       larger samples at these high redshifts.
      %3
      \item We find an anticorrelation between Ly$\upalpha_{\rm{em}}$ and the
       impact parameter $b$, in agreement with previous results at
       high-redshift. 
      %4
      \item To assess the relative importance of LIS and HIS absorptions, we
       compute the C{\sc ii}/C{\sc iv} equivalent width line ratio and found
       that it correlates with impact parameter $b$, stellar mass and star
       formation rate. C{\sc ii}/C{\sc iv} $W_{0}$ line ratio is higher at
       small separations, mainly detected in star-forming galaxies
       ($\langle z \rangle \sim 2.6$) with low stellar masses and low star
       formation rates. Conversely, small C{\sc ii}/C{\sc iv} line ratios are
       defined by stronger C{\sc iv} line absorption compared to C{\sc ii}
       and observed at large separations in star-forming galaxies with higher
       stellar masses and star formation rates.
   \end{enumerate}

The results presented here provide observational evidence consistent with a
scenario where star-forming galaxies at $\langle z \rangle \sim 2.6$ possess
a multi-phase CGM where LIS metal absorptions are produced by denser gas,
that is more extended in these star-forming galaxies. Our results suggest
that galaxies with higher star formation rates and large stellar masses have
stronger ionizing fluxes, able to ionize gas at larger distances and/or are
capable to sweep out the highly ionized gas(traced by C\,{\sc iv}) farther
away from the galaxy compared with less massive and less star-forming
galaxies. Subsequently, star-forming galaxies with low SFR and low stellar
mass show larger reservoirs of cold gas as probed by their C{\sc ii}, C
{\sc iv} and their C{\sc ii}/C{\sc iv} equivalent width line ratio. These
large reservoirs of cold gas could be funnelled into the galaxies and
eventually provide the necessary fuel to sustain star formation activity.
Recently, \cite{WangS22} has demonstrated that large-scale environment
modulates star-formation by regulating the way in which galaxies breathe
in/out (accrete/expell) material by exchanging material within the CGM. This
process is synchronized with star-formation rate events occurring within a
galaxy, and is related to maxima (minima) of SFR associated with a previous
decrease(increase) in the cold circumgalactic gas phase, that halts a further
rise(decline), and leads to a fall (rise) of the star formation rate in later
stages. Our results from C{\sc ii}-peaks and C{\sc iv}-troughs detected in
low-mass galaxies (low SFRs) and high-mass galaxies (high SFRs) could be
interpreted as the snapshots of these two different stages of the complex
interplay between the ISM/CGM/IGM in which galaxies exchange material.
However, we emphasize that high resolution observations are required to
detect outflows expelling out material or inflows accreting
cold material back to the galaxy as recycled material. We highlight that
although stacking increases the S/N of the background \textit
{spotlights}, allowing to detect the faint signal produced by gas in the CGM,
it also smears out the information about the spatial, kinematic and
ionization properties of the CGM. Thus we reiterate the necessity to perform
deeper observations with higher resolutions at this and higher redshift of
multiple LIS and HIS metal lines to provide better constraints on the
properties of the multiphase CGM. It is very likely that future and ongoing observations
from large and deep spectroscopic surveys such as VANDELS \citep
{McLure18,Pentericci18}, zCOSMOS \citep{Lilly07,Lilly09}, VVDS\citep
{LeFevre13b} or DEIMOS10K \citep{Hasinger18} and Integral Field Unit
observations from MUSE/KMOS/JWST will help us understand more about the
multiphase nature of the CGM. Future work should point towards a resolved
view of the velocity field of the in-falling/out-flowing gas of galaxies at
redshift $z > 2$.

%%%%%%%%%%%%%%%%%%%%%%%%%%%%%%%%%%%%%%%%%%%%%%%%%%%%%%%%%%%%%%%%%%%%%%%%%%%%%%%
%%%%%%%%%%%%%%%%%%%%%%%%%%%%%%%%%%CONCLUSIONS%%%%%%%%%%%%%%%%%%%%%%%%%%%%%%%%%%
%%%%%%%%%%%%%%%%%%%%%%%%%%%%%%%%%%%%%%%%%%%%%%%%%%%%%%%%%%%%%%%%%%%%%%%%%%%%%%%

%%%%%%%%%%%%%%%%%%%%%%%%%%%%%%%%%%%%%%%%%%%%%%%%%%%%%%%%%%%%%%%%%%%%%%%%%%%%%%%
%%%%%%%%%%%%%%%%%%%%%%%%%%%%%%%%ACKNOWLEDGMENTS%%%%%%%%%%%%%%%%%%%%%%%%%%%%%%%%
%%%%%%%%%%%%%%%%%%%%%%%%%%%%%%%%%%%%%%%%%%%%%%%%%%%%%%%%%%%%%%%%%%%%%%%%%%%%%%%

\begin{acknowledgements}
        We thank an anonymous referee for constructive comments and
        suggestions that improved the manuscript. Based on data obtained with
        the European Southern Observatory Very Large Telescope, Paranal,
        Chile, under Large Programs 175.A-0839, 177.A-0837, and 185.A-0791.
        This work is based on data products made available at the CESAM data
        center, Laboratoire d'Astrophysique de Marseille. This work partly
        uses observations obtained with MegaPrime/MegaCam, a joint project of
        CFHT and CEA/DAPNIA, at the Canada-France-Hawaii Telescope
        (CFHT) which is operated by the National Research Council (NRC) of
        Canada, the Institut National des Sciences de l'Univers of the Centre
        National de la Recherche Scientifique (CNRS) of France, and the
        University of Hawaii. This work is based in part on data products
        produced at TERAPIX and the Canadian Astronomy Data Centre as part of
        the Canada-France-Hawaii Telescope Legacy Survey, a collaborative
        project of NRC and CNRS. HMH acknowledge partial support from
        National Fund for Scientific and Technological Research of Chile
        (\emph{Fondecyt}) through grants no. 1171710 \& 1150216. E.I.\
        acknowledges partial support from FONDECYT through grant
        N$^\circ$\,1171710. MA acknowledges support from FONDECYT grant
        1211951, "ANID+PCI+INSTITUTO MAX PLANCK DE ASTRONOMIA MPG 190030"
        and "ANID+PCI+REDES 190194". RA acknowledges support from ANID
        FONDECYT Regular Grant 1202007. We thank ESO staff for their support
        for the VUDS survey, particularly the Paranal staff conducting the
        observations and Marina Rejkuba and the ESO user support group in
        Garching. This research used Astropy,\footnote
        {http://www.astropy.org} a community-developed core Python package
        for Astronomy \citep{astropy:2013, astropy:2018}, Numpy \citep
        {numpy:2020}, Scipy \citep{SciPy:2020} and Matplotlib \citep
        {matplotlib:2007}
\end{acknowledgements}
%%%%%%%%%%%%%%%%%%%%%%%%%%%%%%%%%%%%%%%%%%%%%%%%%%%%%%%%%%%%%%%%%%%%%%%%%%%%%%%
%%%%%%%%%%%%%%%%%%%%%%%%%%%%%%%%ACKNOWLEDGMENTS%%%%%%%%%%%%%%%%%%%%%%%%%%%%%%%%
%%%%%%%%%%%%%%%%%%%%%%%%%%%%%%%%%%%%%%%%%%%%%%%%%%%%%%%%%%%%%%%%%%%%%%%%%%%%%%%

%%%%%%%%%%%%%%%%%%%%%%%%%%%%%%%%%%%%%%%%%%%%%%%%%%%%%%%%%%%%%%%%%%%%%%%%%%%%%%%
%%%%%%%%%%%%%%%%%%%%%%%%%%%%%%%%%%REFERENCES%%%%%%%%%%%%%%%%%%%%%%%%%%%%%%%%%%%
%%%%%%%%%%%%%%%%%%%%%%%%%%%%%%%%%%%%%%%%%%%%%%%%%%%%%%%%%%%%%%%%%%%%%%%%%%%%%%%
\bibliographystyle{aa} % style aa.bst
\bibliography{References_2.bib} % your references Yourfile.bib
%%%%%%%%%%%%%%%%%%%%%%%%%%%%%%%%%%%%%%%%%%%%%%%%%%%%%%%%%%%%%%%%%%%%%%%%%%%%%%%
%%%%%%%%%%%%%%%%%%%%%%%%%%%%%%%%%%REFERENCES%%%%%%%%%%%%%%%%%%%%%%%%%%%%%%%%%%%
%%%%%%%%%%%%%%%%%%%%%%%%%%%%%%%%%%%%%%%%%%%%%%%%%%%%%%%%%%%%%%%%%%%%%%%%%%%%%%%

%%%%%%%%%%%%%%%%%%%%%%%%%%%%%%%%%%%%%%%%%%%%%%%%%%%%%%%%%%%%%%%%%%%%%%%%%%%%%%%
%%%%%%%%%%%%%%%%%%%%%%%%%%%%%%%%%%%%Appendix%%%%%%%%%%%%%%%%%%%%%%%%%%%%%%%%%%%
%%%%%%%%%%%%%%%%%%%%%%%%%%%%%%%%%%%%%%%%%%%%%%%%%%%%%%%%%%%%%%%%%%%%%%%%%%%%%%%
\begin{appendix} \label{sec:Appendix} %First appendix
\section{Supplementary tables.} \label{sec:AppendixTbl}
%\appendix 
\FloatBarrier
%%%%%%%%%%%%%%%%%%%%%%%%%%%%%%%%%TBL-FG-BG-STAT%%%%%%%%%%%%%%%%%%%%%%%%%%%%%%%%
%%%%%%%%%%%%%%%%%%%%%%%%%%%%%%%%%%%%%%STM%%%%%%%%%%%%%%%%%%%%%%%%%%%%%%%%%%%%%%
\begin{table}[h]
\centering
\caption{\label{tbl:PairsStat-STM}
\textit{Foreground-background} (\textit{fg-bg}) galaxy pair statistics 
according to their stellar mass (log[M$_{\bigstar}$/M$_{\odot}$]).}
\begin{tabular}{cccccc}
  \\
  \hline\hline
  \multicolumn{1}{c}{ID} &
  \multicolumn{1}{c}{N} &
  \multicolumn{1}{c}{$\langle$ log[M$_{\bigstar}$/M$_{\odot}$] $\rangle$} &
  \multicolumn{1}{c}{$\langle b \rangle$ (kpc)} & 
  \multicolumn{1}{c}{$\langle \rm{z}_{\rm{fg}}\rangle$} &
  \multicolumn{1}{c}{$\langle \rm{z}_{\rm{bg}}\rangle$} \\
  \multicolumn{1}{c}{c1} & 
  \multicolumn{1}{c}{c2} &
  \multicolumn{1}{c}{c3} &
  \multicolumn{1}{c}{c4} &
  \multicolumn{1}{c}{c5} &
  \multicolumn{1}{c}{c6} \\
  \hline\hline  
ALL & 238 &  9.73 $\pm$ 0.40 & 125.05 & 2.60 & 3.04\\
S1  &  61 &  9.26 $\pm$ 0.18 & 125.12 & 2.48 & 2.87 \\
S2  &  63 &  9.60 $\pm$ 0.07 & 119.63 & 2.65 & 3.10 \\
S3  &  65 &  9.87 $\pm$ 0.09 & 130.42 & 2.58 & 3.04 \\
S4  &  49 & 10.31 $\pm$ 0.24 & 124.83 & 2.71 & 3.19 \\
  \hline\hline
\end{tabular}
\tablefoot{
Column 1: Sample ID; 
Column 2: number of galaxies per sample; 
Column 3: mean stellar mass;
Column 4: mean impact parameter in kpc;
Column 5: mean redshift of the \textit{foreground} galaxies sample, and
Column 6: mean redshift of the \textit{background} galaxies sample.}
\end{table}
%%%%%%%%%%%%%%%%%%%%%%%%%%%%%%%%%%%%%%STM%%%%%%%%%%%%%%%%%%%%%%%%%%%%%%%%%%%%%%
%%%%%%%%%%%%%%%%%%%%%%%%%%%%%%%%%TBL-FG-BG-STAT%%%%%%%%%%%%%%%%%%%%%%%%%%%%%%%%

%%%%%%%%%%%%%%%%%%%%%%%%%%%%%%%%%TBL-FG-BG-STAT%%%%%%%%%%%%%%%%%%%%%%%%%%%%%%%%
%%%%%%%%%%%%%%%%%%%%%%%%%%%%%%%%%%%%%%SFR%%%%%%%%%%%%%%%%%%%%%%%%%%%%%%%%%%%%%%
\begin{table}[h]
\centering
\caption{\label{tbl:PairsStat-SFR}
\textit{Foreground-background} (\textit{fg-bg}) galaxy pair statistics 
according to their star formation rate (log[(M$_{\odot}$yr$^{-1})]$).}
\begin{tabular}{cccccc}
  \\
  \hline\hline
  \multicolumn{1}{c}{ID} &
  \multicolumn{1}{c}{N} &
  \multicolumn{1}{c}{$\langle$ log[SFR/(M$_{\odot}$yr$^{-1})]$ $\rangle$ }&
  \multicolumn{1}{c}{$\langle b \rangle$ (kpc)} & 
  \multicolumn{1}{c}{$\langle \rm{z}_{\rm{fg}}\rangle$} &
  \multicolumn{1}{c}{$\langle \rm{z}_{\rm{bg}}\rangle$} \\
  \multicolumn{1}{c}{c1} & 
  \multicolumn{1}{c}{c2} &
  \multicolumn{1}{c}{c3} &
  \multicolumn{1}{c}{c4} &
  \multicolumn{1}{c}{c5} &
  \multicolumn{1}{c}{c6} \\
  \hline\hline  
ALL & 238 & 1.37 $\pm$  0.39 & 125.10 & 2.60 & 3.04 \\
S1  &  60 & 0.91 $\pm$  0.39 & 117.92 & 2.56 & 2.98 \\
S2  &  62 & 1.21 $\pm$  0.06 & 128.63 & 2.64 & 3.08 \\
S3  &  59 & 1.49 $\pm$  0.09 & 124.50 & 2.54 & 3.02 \\
S4  &  59 & 1.92 $\pm$  0.20 & 129.21 & 2.67 & 3.08 \\
  \hline\hline
\end{tabular}
\tablefoot{
Column 1: Sample ID; 
Column 2: number of galaxies per sample; 
Column 3: mean star formation rate (log[SFR/(M$_{\odot}$yr$^{-1})]$);
Column 4: mean impact parameter in kpc;
Column 5: mean redshift of the \textit{foreground} galaxies sample, and
Column 6: mean redshift of the \textit{background} galaxies sample.}
\end{table}
%%%%%%%%%%%%%%%%%%%%%%%%%%%%%%%%%%%%%%SFR%%%%%%%%%%%%%%%%%%%%%%%%%%%%%%%%%%%%%%
%%%%%%%%%%%%%%%%%%%%%%%%%%%%%%%%%TBL-FG-BG-STAT%%%%%%%%%%%%%%%%%%%%%%%%%%%%%%%%

%%%%%%%%%%%%%%%%%%%%%%%%%%%%%%%%%TBL-FG-BG-STAT%%%%%%%%%%%%%%%%%%%%%%%%%%%%%%%%
%%%%%%%%%%%%%%%%%%%%%%%%%%%%%%%%%%%%%%PHI%%%%%%%%%%%%%%%%%%%%%%%%%%%%%%%%%%%%%%
\begin{table}[h]
\centering
\caption{\label{tbl:PairsStat-PHI}
\textit{Foreground-background} (\textit{fg-bg}) galaxy pair statistics 
according to their azimuthal angle ($\phi$).}
\begin{tabular}{cccccc}
  \\
  \hline\hline
  \multicolumn{1}{c}{ID} &
  \multicolumn{1}{c}{N} &
  \multicolumn{1}{c}{$\langle$ $\phi \rangle$ ($^{\circ}$)}&
  \multicolumn{1}{c}{$\langle b \rangle$ (kpc)} & 
  \multicolumn{1}{c}{$\langle \rm{z}_{\rm{fg}}\rangle$} &
  \multicolumn{1}{c}{$\langle \rm{z}_{\rm{bg}}\rangle$} \\
  \multicolumn{1}{c}{c1} & 
  \multicolumn{1}{c}{c2} &
  \multicolumn{1}{c}{c3} &
  \multicolumn{1}{c}{c4} &
  \multicolumn{1}{c}{c5} &
  \multicolumn{1}{c}{c6} \\
  \hline\hline  
ALL &  97 & 31.34 $\pm$ 18.12 & 121.54 & 2.65 & 3.08 \\
S1  &  60 & 16.90 $\pm$  6.20 & 107.74 & 2.69 & 3.15 \\
S2  &  62 & 44.92 $\pm$  6.74 & 132.53 & 2.55 & 2.92 \\
S3  &  59 & 75.13 $\pm$  9.10 & 118.75 & 2.70 & 3.16 \\
  \hline\hline
\end{tabular}
\tablefoot{
Column 1: Sample ID; 
Column 2: number of galaxies per sample; 
Column 3: mean azimuthal angle ($\phi \, ^{\circ}$);
Column 4: mean impact parameter in kpc;
Column 5: mean redshift of the \textit{foreground} galaxies sample, and
Column 6: mean redshift of the \textit{background} galaxies sample.}
\end{table}
%%%%%%%%%%%%%%%%%%%%%%%%%%%%%%%%%%%%%%PHI%%%%%%%%%%%%%%%%%%%%%%%%%%%%%%%%%%%%%%
%%%%%%%%%%%%%%%%%%%%%%%%%%%%%%%%%TBL-FG-BG-STAT%%%%%%%%%%%%%%%%%%%%%%%%%%%%%%%%

%%%%%%%%%%%%%%%%%%%%%%%%%%%%%%%%%TBL-FG-BG-STAT%%%%%%%%%%%%%%%%%%%%%%%%%%%%%%%%
%%%%%%%%%%%%%%%%%%%%%%%%%%%%%%%%%%%%%REFF%%%%%%%%%%%%%%%%%%%%%%%%%%%%%%%%%%%%%%
\begin{table}[h]
\centering
\caption{\label{tbl:PairsStat-REFF}
\textit{Foreground-background} (\textit{fg-bg}) galaxy pair statistics 
according to their effective radius (r$_{\rm{eff}}$ \, kpc).}
\begin{tabular}{cccccc}
  \\
  \hline\hline
  \multicolumn{1}{c}{ID} &
  \multicolumn{1}{c}{N} &
  \multicolumn{1}{c}{$\langle$ r$_{\rm{eff}}$ $\rangle$ (kpc) }&
  \multicolumn{1}{c}{$\langle b \rangle$ (kpc)} & 
  \multicolumn{1}{c}{$\langle \rm{z}_{\rm{fg}}\rangle$} &
  \multicolumn{1}{c}{$\langle \rm{z}_{\rm{bg}}\rangle$} \\
  \multicolumn{1}{c}{c1} & 
  \multicolumn{1}{c}{c2} &
  \multicolumn{1}{c}{c3} &
  \multicolumn{1}{c}{c4} &
  \multicolumn{1}{c}{c5} &
  \multicolumn{1}{c}{c6} \\
  \hline\hline  
ALL &  97 &  8.94 $\pm$ 0.49 & 121.54 & 2.65 & 3.08 \\
S1  &  31 &  2.70 $\pm$ 0.98 & 125.92 & 2.67 & 3.16 \\
S2  &  33 &  6.87 $\pm$ 1.40 & 120.75 & 2.63 & 3.02 \\
S3  &  33 & 16.88 $\pm$ 7.39 & 118.19 & 2.66 & 3.07 \\
  \hline\hline
\end{tabular}
\tablefoot{
Column 1: Sample ID; 
Column 2: number of galaxies per sample; 
Column 3: mean effective radius (r$_{\rm{eff}}$ \, kpc);
Column 4: mean impact parameter in kpc;
Column 5: mean redshift of the \textit{foreground} galaxies sample, and
Column 6: mean redshift of the \textit{background} galaxies sample.}
\end{table}
%%%%%%%%%%%%%%%%%%%%%%%%%%%%%%%%%%%%%REFF%%%%%%%%%%%%%%%%%%%%%%%%%%%%%%%%%%%%%%
%%%%%%%%%%%%%%%%%%%%%%%%%%%%%%%%%TBL-FG-BG-STAT%%%%%%%%%%%%%%%%%%%%%%%%%%%%%%%%

%\section{Supplementary figures.} \label{sec:AppendixFig}
\end{appendix}
%%%%%%%%%%%%%%%%%%%%%%%%%%%%%%%%%%%%%%%%%%%%%%%%%%%%%%%%%%%%%%%%%%%%%%%%%%%%%%%
%%%%%%%%%%%%%%%%%%%%%%%%%%%%%%%%%%%%Appendix%%%%%%%%%%%%%%%%%%%%%%%%%%%%%%%%%%%
%%%%%%%%%%%%%%%%%%%%%%%%%%%%%%%%%%%%%%%%%%%%%%%%%%%%%%%%%%%%%%%%%%%%%%%%%%%%%%%
\end{document}
